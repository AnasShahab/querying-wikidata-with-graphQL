\documentclass[12 pt, a4paper]{report}
%\documentclass[runningheads]{llncs}
\usepackage[T1]{fontenc}
\usepackage{mathptmx}
%\usepackage{tabularx}
%\usepackage{array}
\usepackage{amsmath,amssymb,amsfonts, amsthm}
\usepackage{rotating}
\usepackage{pifont}% http://ctan.org/pkg/pifont
\newcommand{\cmark}{\ding{51}}
\newcommand{\xmark}{\ding{55}}
\usepackage{tablefootnote}
\theoremstyle{definition}
\newtheorem{definition}{Definition}[section]

%\newcommand{\citeauthor}[1]{\citeauthor*{#1}}

\usepackage{setspace}
%\usepackage[top=1 in,bottom=1 in,left=3.2 cm,right=2.6 cm]{geometry}
\usepackage[utf8]{inputenc}
\usepackage{fullpage}
\usepackage{graphicx}
%\renewcommand{\baselinestretch}{2}
%\renewcommand{\thesection}{\arabic{section}}
%\raggedbottom
%\usepackage{cite}
%\usepackage{biblatex}
\usepackage[sort&compress,numbers]{natbib}
\usepackage[hidelinks]{hyperref}
\usepackage[nottoc]{tocbibind}
\usepackage{rotating}
\usepackage{hyperref}
\usepackage{lipsum}
\usepackage{xcolor}
%\usepackage{afterpage}
\usepackage{parskip} 
%removes indentation but gives some spacing ebtween parae the
\usepackage{todonotes}
\usepackage[toc,acronym]{glossaries}

\usepackage{booktabs}
\usepackage{rotating}
%\usepackage{minted}
\usepackage{listings}
%\usepackage{float}
%\usepackage{caption}
\definecolor{codegreen}{rgb}{0,0.6,0}
\definecolor{codegray}{rgb}{0.5,0.5,0.5}
\definecolor{codepurple}{rgb}{0.58,0,0.82}
\definecolor{backcolour}{rgb}{0.95,0.95,0.92}


\lstdefinestyle{mystyle}{
    backgroundcolor=\color{backcolour},   
%    commentstyle=\color{codegreen},
    keywordstyle=\color{magenta},
    numberstyle=\tiny\color{codegray},
%    stringstyle=\color{codepurple},
    basicstyle=\ttfamily\scriptsize,
    breakatwhitespace=false,         
    breaklines=true,                 
    captionpos=b,                    
    keepspaces=true,                 
    numbers=none,                    
    numbersep=5pt,                  
    showspaces=false,                
    showstringspaces=false,
    showtabs=false,                  
    tabsize=2,
    escapechar=\%,
    framexleftmargin=0.25mm,
    frame=single,
%    float=H,
%    escapeinside=``,
    rulecolor=\color{black}
%    moredelim=[s][\color{purple}]{<}{>}
%    alsoletter={<},  moredelim=[s][\color{blue}\bfseries]{<}{>},
%    moredelim=[is][\color{blue}\bfseries\ttfamily]{<}{>}
%    moredelim=[is][\color{blue\bfseries]{[}{]},  
%    flexiblecolumns=true	
}

%\lstdefinestyle{mystyle2}{
%    backgroundcolor=\color{backcolour},   
%    commentstyle=\color{codegreen},
%    keywordstyle=\color{magenta},
%    numberstyle=\tiny\color{codegray},
%    stringstyle=\color{codepurple},
%    basicstyle=\ttfamily\large,
%    breakatwhitespace=false,         
%    breaklines=true,                 
%    captionpos=b,                    
%    keepspaces=true,                 
%    numbers=none,                    
%    numbersep=5pt,                  
%    showspaces=false,                
%    showstringspaces=false,
%    showtabs=false,                  
%    tabsize=2,
%    escapechar=\%,
%    framexleftmargin=0.25mm,
%    frame=single,
%    rulecolor=\color{black}
%    flexiblecolumns=true	
%}

\lstset{style=mystyle}


\lstdefinelanguage{SPARQL}{
  keywords={BASE, PREFIX, SELECT, WHERE, ChemicalCompound_GET, Cat_GET, OPTIONAL, query},
  keywordstyle=\color{red}\bfseries,
  ndkeywords={FILTER, LANG, LIMIT, OFFSET, first, offset, limit, SERVICE,@optional, lang, COUNT, VALUES},
  ndkeywordstyle=\color{blue}\bfseries,
  moredelim=[s][\color{blue}]{<}{>},
%  alsoletter={\$}, moredelim=[s][\color{red}]{[},
%  alsoletter={[},  moredelim=[s][\color{red}\bfseries]{[}{]},  
%  identifierstyle=\color{black},
  sensitive=true
%  morecomment=[l]{\#},
%  morestring=[b]',
%  morestring=[b]",
%  alsodigit={-},
%  basicstyle=\ttfamily,  
%  stringstyle=\color{red}\itshape,
%  commentstyle=\color{gray}\ttfamily
%  literate=
%    {->}{{\(\rightarrow\)}}2
%    {<=}{{\(\leq\)}}2
%    {>=}{{\(\geq\)}}2
%    {|}{{\(\mid\)}}1
}

\lstdefinelanguage{GraphQL}{
  keywords={query,mutation,subscription,fragment,on, type, data, @context, subject, predicate, object, graph, results, ChemicalElement_GET, extensions, errors},
  keywordstyle=\color{red}\bfseries,
  keywords=[3]{String, Int, Float, Boolean, limit, ChemicalElement, first, offset, if, Quad, bindings, lang},
  keywordstyle=[3]\color{blue}\bfseries,
  keywords=[2]{chemicalElement, name, id, label, chemicalFormula, boilingPoint, meltingPoint, density, discoverer, placeBirth, country, wd, wdt, instance,_id, Human, City, SovereignState, _type, @vocab, cat, chemicalCompound, color, rgb, JohannSebastianBach, hasFather, fatherOf, hasMother, cathedral, place, Paris, Cat, ChemicalCompound, Color, Airport, connectingLine, coordinateLocation, placeServed},
  keywordstyle=[2]\color{codegreen}\bfseries,
%  morekeywordstyle=\color{green}\bfseries,
  moredelim=[s][\color{blue}]{"http}{"},
%  moredelim=[s][\color{purple}]{<}{>},
%  alsoletter={\$}, moredelim=[s][\color{red}]{[},
%  alsoletter={[},  moredelim=[s][\color{red}\bfseries]{[}{]},  
%  identifierstyle=\color{black},
  sensitive=true
%  morecomment=[l]{\#},
%  morestring=[b]',
%  morestring=[b]",
%  alsodigit={-},
%  basicstyle=\ttfamily,  
%  stringstyle=\color{red}\itshape,
%  commentstyle=\color{gray}\ttfamily
%  literate=
%    {->}{{\(\rightarrow\)}}2
%    {<=}{{\(\leq\)}}2
%    {>=}{{\(\geq\)}}2
%    {|}{{\(\mid\)}}1
}

\lstdefinelanguage{GraphQLException}{
  keywords={query,mutation,subscription,fragment,on, data, @context, results, chemicalElement, bindings},
  keywordstyle=\color{red}\bfseries,
  keywords=[3]{String, Int, Float, Boolean, limit, ChemicalElement, first, offset, if, Quad}, keywordstyle=[3]\color{blue}\bfseries,
  keywords=[2]{name, chemicalFormula, boilingPoint, meltingPoint, density, input, variables, type, subject, predicate, object, graph, value},
  keywordstyle=[2]\color{codegreen}\bfseries,
%  morekeywordstyle=\color{green}\bfseries,
  moredelim=[s][\color{blue}]{"http}{"},
  sensitive=true
}

\lstdefinelanguage{GraphQLException2}{
  keywords={Quad, chemicalElement},
  keywordstyle=\color{red}\bfseries,
%  keywords=[3]{String, Int, Float, Boolean, limit, ChemicalElement, first, offset, if, bindings},
%  keywordstyle=[3]\color{blue}\bfseries,
  keywords=[2]{termType, value, subject, predicate, object, graph, type, name, chemicalFormula, boilingPoint},
  keywordstyle=[2]\color{codegreen}\bfseries,
%  morekeywordstyle=\color{green}\bfseries,
  moredelim=[s][\color{blue}]{"http}{"},
  sensitive=true
}

\lstdefinelanguage{GraphQLException3}{
%  keywords={Quad, chemicalElement},
%  keywordstyle=\color{red}\bfseries,
%  keywords=[3]{String, Int, Float, Boolean, limit, ChemicalElement, first, offset, if, bindings},
%  keywordstyle=[3]\color{blue}\bfseries,
  keywords=[2]{name, schema, server, services},
  keywordstyle=[2]\color{codegreen}\bfseries,
%  morekeywordstyle=\color{green}\bfseries,
  moredelim=[s][\color{blue}]{"http}{"},
  sensitive=true
}

%\lstset{language=GraphQL}
\setlength {\marginparwidth }{2cm} 
\newcommand{\lukas}[1]{\todo[color=green,inline]{#1}}
\newcommand{\anas}[1]{\todo[color=red,inline]{#1}}

\newcommand{\blankpage}{
\newpage
\thispagestyle{empty}
\addtocounter{page}{-1}
\mbox{}
\newpage
}

%\newcommand\blankpage{%
%    \null
%    \thispagestyle{empty}%
%    \addtocounter{page}{-1}%
%    \newpage}

%\def\keywords{\vspace{.5em}
%{\textit{Keywords}:\,\relax%
%}}
%\def\endkeywords{\par}

%\providecommand{\keywords}[1]
%{
%  \small	
%  \textbf{\textit{Keywords---}} #1
%}

\counterwithout{footnote}{chapter}



%\newglossaryentry{wd}
{
    name=wd,
    description={Is the prefix for the namespace - http://www.wikidata.org/entity used for items in Wikidata}
}

\newglossaryentry{wdt}
{
    name=wdt,
    description={Is the prefix for the namespace - http://www.wikidata.org/prop/direct/ used for properties in Wikidata}
}


\newglossaryentry{rdf}
{
    name=rdf,
    description={Is the prefix for the namespace - http://www.w3.org/1999/02/22-rdf-syntax-ns\# used in RDF}
}

\newglossaryentry{rdfs}
{
    name=rdfs,
    description={Is the for the namespace - http://www.w3.org/2000/01/rdf-schema\# where the core vocabulary of RDF is defined}
}


\newglossaryentry{xsd}
{
    name=xsd,
    description={Is the prefix for the namespace - http://www.w3.org/2001/XMLSchema\# which is the XML Schema that defines the datatypes}
}

\newglossaryentry{Q560}
{
    name=Q560,
    description={Is the ID for the item "Helium" in Wikidata}
}

\newglossaryentry{Q11344}
{
    name=Q11344,
    description={Is the ID for the item "chemical element" in Wikidata}
}

\newglossaryentry{Q298581}
{
    name=Q298581,
    description={Is the ID for the item "Pierre Janssen" in Wikidata}
}

\newglossaryentry{Q90}
{
    name=Q90,
    description={Is the ID for the item "Paris" in Wikidata}
}

\newglossaryentry{Q142}
{
    name=Q142,
    description={Is the ID for the item "France" in Wikidata}
}

\newglossaryentry{P31}
{
    name=P31,
    description={Is the ID for the property "instance of" in Wikidata}
}

\newglossaryentry{P2102}
{
    name=P2102,
    description={Is the ID for the property "boiling point" in Wikidata}
}

\newglossaryentry{P274}
{
    name=P274,
    description={Is the ID for the property "chemical formula" in Wikidata}
}

\newglossaryentry{P2101}
{
    name=P2101,
    description={Is the ID for the property "melting point" in Wikidata}
}

\newglossaryentry{P2054}
{
    name=P2054,
    description={Is the ID for the property "density" in Wikidata}
}

\newglossaryentry{P61}
{
    name=P61,
    description={Is the ID for the property "discoverer/inventor" in Wikidata}
}

\newglossaryentry{P625}
{
    name=P625,
    description={Is the ID for the property "place of birth" in Wikidata}
}

%\newacronym{RDF}{Resource Description Framework}

\newacronym{URI}{URI}{Uniform Resource Identifier}

\newacronym{IRI}{IRI}{Internationalized Resource identifier}

\newacronym{URL}{URL}{Uniform Resource Locator}

\newacronym{BCP47}{BCP47}{Best Current Practice 47}

\newacronym{SPARQL}{SPARQL}{SPARQL Protocol and RDF Query Language}

\newacronym{W3C}{W3C}{World Wide Web Consortium}




\onehalfspacing
\begin{document}

%\maketitle
\sloppy
\begin{titlepage}

\includegraphics[width=0.4\textwidth]{logo.jpg}

\LARGE
\hrule
\vspace*{0.20cm}
Faculty of  Computer Science
\vspace*{0.15cm}
\hrule

    \begin{center}
        \vspace*{1cm}       
            
        \Huge
        \textbf{Master Thesis}
            
        \vspace{0.5cm}
        \LARGE
        Querying Wikidata with GraphQL
            
        \vspace{2cm}
        
        
        
	 \begin{flushleft}
		Anas Shahab \\
    		Master Computational Logic \\
    		Matriculation number: 123456789
    		
    		\vspace{1cm}
    		
    		Supervisors: \\
    		Prof. Dr. Markus Kr{\"otzsch} \\
    		Another Supervisor Name
    		\vspace{0.5cm}
    		
    		Tutor: \\
    		Dipl.-Inf. Lukas Gerlach
    		
    		\vspace{0.5cm}
    		
    		Chair of Knowledge-Based Systems
    		
    		\vspace{1cm}
    		
    		Submission Date:
    \end{flushleft}
                  
         
            
    \end{center}
\end{titlepage}
\blankpage
%\afterpage{\blankpage}

%\thispagestyle{empty}
\pagenumbering{roman}
\chapter*{Declaration of originality}
%\section*{Declaration of originality}
I hereby declare that I have written this Thesis on my own accord and any participation of others has been acknowledged. I have clearly marked all references to existing work. I have not submitted this work partly or as a whole anywhere else. \\

Dresden, \textcolor{red}{XX.XX.2023} \\
\rule{150 px}{0.5 px} \\
\textcolor{red}{(signature)}
\blankpage
%\afterpage{\blankpage}

%\thispagestyle{empty}
\chapter*{Acknowledgements}
%\section*{Acknowledgements}
\textcolor{red}{\lipsum[1]}
\blankpage

%\begin{abstract}
%\lipsum[1]
%\\[0.5 cm]
%\centerline{\keywords{x, y, z}}
%
%\end{abstract}

\chapter*{Abstract}
%\section*{Abstract}
A Knowledge graph is a collection of data that is stored in the form of RDF graphs. It represents knowledge that conveys information about the real world. Knowledge graphs can be modelled using different graphs, the most commonly one being directed-edge labelled graphs. The nodes of the graph represent entities while the edges represent the relationships that exists between the entities. Knowledge graphs have several uses both in commercial and research domains. There are used widely by large companies like Google and Amazon in their applications. Some of the popular uses of knowledge graphs in the field of research are in data mining and machine learning.
\lukas{Abstract should be mostly self contained; RDF was not introduced for example (spelling it out once would be enough)}

Wikidata is a free and publicly available knowledge graph created at Wikimedia Deutschland. It is popular within the research community and has a has a wide range of applications. Many large organizations such as Apple and Amazon use Wikidata for their applications. Since it has a large amount of real world data it is used for collecting data in machine learning algorithms.

Wikidata is built on top of RDF but it has some key differences with RDF data model. RDF is a framework that represents information in the form of Linked Data. Both RDF and RDFS provide vocabulary that defines resources and properties. RDF graphs are directed-edge labelled graphs composed of a set of triples - <subject, predicate, object>. A syntactic representation is needed to exchange RDF graphs. There are several formats for the representation such as N-Triples and Turtle.

SPARQL is a protocol and query language for RDF. It is based on matching graph patterns and can query data source structured in RDF format. It supports a variety of query types such as SELECT and CONSTRUCT. 


\anas{complete this}
\lukas{need to see this again when finished}



%\pagenumbering{roman}
\tableofcontents
\newpage
%\renewcommand*{\lstlistingname}{List of XYZ}
\listoffigures
\newpage
\listoftables
\newpage
\lstlistoflistings
%\addcontentsline{toc}{chapters}{\lstlistlistingname}
\newpage
\pagebreak

%\doublespacing
\pagenumbering{arabic}

\chapter{Introduction}
%\section{Introduction}

The term "knowledge graph" gained popularity in 2012 when Google launched its own Google Knowledge Graph. A knowledge graph is a collection of data represented as a graph. There are many ways of modelling data as a graph. The most commonly used ones are directed edge-labelled graphs, heterogeneous graphs, property graphs and graph dataset \cite{Hogan2021}. In this report, we consider directed edge-labelled graphs for modelling knowledge graphs.

A directed edge-labelled graph, also known as a multi-relational graph, consists of a set of nodes and a set of directed labelled edges \cite{Hogan2021}. The information represented in knowledge graphs conveys knowledge of the real world, where the nodes represent entities of interest and the directed edges represent the many different binary relations between those entities. Entities are real world objects and abstract concepts. An object is a physical item in the real world such a university (e.g., Technical University Dresden) and a planet (e.g., Earth). A concept on the other hand refers to general categories of objects such as chemical element and philosopher. 

For example, "Helium is a type of chemical element that has the chemical formula He", 
is a simple piece of information that can be represented using a directed edge-labelled graph. The nodes of the graph would represent the entities "Helium", "chemical element" and "He". There would be two labelled edges, one pointing from the "Helium" node to the "chemical element" node, and the other from the "Helium" node to the "He" node. These edges would represent the relations "instance of" and "chemical formula" respectively. Figure 1 shows such a direct-edge labelled graph. We provide a more complex example later in Chapter 2.

\begin{figure}[h]
  \centering
  \includegraphics[width=0.75 \linewidth]{images/knowledge_graph.drawio.pdf}
  \caption{An RDF graph with Subject and Object nodes connected via a predicate edge}
  \label{fig:figure 1}
\end{figure}

Resource Description Framework (RDF) is a standard framework that models the Semantic Web. It bases its data model on directed edge-labelled graphs and is composed of a set of triples or statements. Each triple consist of a subject node, predicate edge and an object node. The subject and object nodes represent the source and destination respectively while the predicate edge connects these two nodes representing a binary relationship between them. RDF is useful for describing and exchanging graphs over the web. 

Knowledge graphs have practical uses in commercial and scientific domains. Many companies such as Amazon, Facebook, Uber, Google, etc., use knowledge graphs for their applications. For example, Google and Bing use their knowledge graphs, Google Knowledge Graph and Satori respectively, to enhance the results in their search engines. In the field of life sciences, various knowledge graphs such as Neurocommons and LinkedLifeData exist that contain biomedical information from different sources \cite{Nickel2015}. 

Depending on the organization or community there are open or enterprise knowledge graphs \cite{Hogan2021}. Open knowledge graphs include Wikidata, DBpedia, Freebase, YAGO, etc. These are available online and freely accessible to the public. On the other hand, enterprise knowledge graphs are used internally within companies and are aimed towards solving their specific use-cases

Wikidata is a free and publicly available knowledge base that can be read and edited by both humans and machines. It is one of the many projects by Wikimedia Foundations such as Wikipedia, Wikibooks, Wikimedia Commons and Wikitionary. Wikidata was created in 2012 at Wikimedia Deutschland by a community of volunteers. These volunteers edit and control all content. As of December 2022, Wikidata has more 23000 active editors /footnote(https://wikidata.wikiscan.org/). Wikidata provides a website where data can be viewed and also edited \cite{Foundationa}. 

One of the original purposes behind the creation of Wikidata was to help its sister projects. Initially, Wikipedia and its sister projects used to maintain their own lists of interlanguage links. This refers to links between Wikipedia articles about the same topic but in different languages. After 2012, these interlanguage links were provisioned via Wikidata. Wikidata is also used to display data shown inside the pages in Wikipedia. The usage of this mainly depends on the language of the Wiki. For instance, in the case of some languages, parts of the Wikipedia pages are created automatically from the data in Wikidata. Others, especially those of smaller languages that are not widely used, use Wikidata to create placeholder pages when an article may not be in their respective language. In 2018, around 59\% of Wikidata information was used in English Wikipedia articles, although mostly for external identifiers and coordinate locations \cite{Wikipedia2017}. 

Wikidata stores information in the form of structured data in a database \cite{Tharani2021}. This is not the case for its sister projects as they contain unstructured data. The information on their web pages is not directly given a structure in the form of tables or lists. Wikidata acts as a central storage for these projects and focuses on providing a structure for their data \cite{Wikidata2014}. Additionally, Wikidata also supports linked data. This means that the data stored can be linked to datasets and databases like Google Books and OmegaWiki. Wikidata can also be used for quality checks against Wikipedia articles. This is useful when information about a specific topic needs to be known and the solution is easily found by querying the knowledge graph, in this case Wikidata.

Moreover, there is also a wide range of commercial and research oriented applications for Wikidata. This is due to the fact that it has a large amount of real world data. For instance, Wikidata has external usages in many large organizations such as Eurowings, Google, Apple and Amazon. This includes tasks such as data integration, authority control, identity providing and data-driven journalism. In the field of research, Wikidata is used for collecting test data for knowledge graph related algorithms and training data for machine learning projects.

SPARQL is a W3C standard query language based on matching graph patterns. It is used to query information from any source that maps its data in RDF format. Wikidata is built on RDF framework and can be can be queried using SPARQL. SPARQL has gained widespread popularity in research and academics. However, in commercial applications like application development, SPARQL faces some potential barriers. This is mainly due to the complex nature of SPARQL queries, and the lack of libraries and frameworks to facilitate its integration in applications. 

GraphQL is an open source query language popular in commercial applications. It was developed by Facebook in 2012 and was made open source later in 2015. It is easy to learn and use, providing syntax that is more human friendly than SPARQL. It is possible to implement GraphQL different programming languages and many libraries exist to support the integration into application development. Queries in GraphQL have a tree like structure, where the root or parent node is the object and the children nodes are the fields for that object. The result obtained have the same shape as the queries, and this implies that we always get back what we expect. This makes GraphQL convenient to use. 

Owing to the simplistic nature of GraphQL queries and its ease of integration in application development, GraphQL is a prospective approach to query RDF data. Consequently, this further opens the scope of using knowledge graphs through querying to retrieve and having a better understanding of the data that lies within them.

In this report, we research on existing mechanisms to query RDF data using GraphQL. We mainly focus on two approaches - GraphQL-LD and HyperGraphQL. Both of these are open source and can be used to query arbitrary knowledge graphs using GraphQL. We show how these two approaches can be used to query Wikidata. We also show how these two compare with each other in terms of functionality and limitations, keeping standard GraphQL features as a benchmark. Lastly, we provide a comparison of the two approaches against some standard SPARQL queries.


The remainder of the report is structured as follows.
\begin{itemize}
	\item In chapter 2, we provide an overview of RDF, Wikidata, SPARQL and GraphQL. We also give a comparison between GraphQL and SPARQL. 
	\item In chapter 3, we show the existing approaches used to query RDF graphs using GraphQL, focusing on GraphQL-LD and HyperGraphQL.
	\item In chapter 4, we aim to provide implementation of GraphQL-LD and HyperGraphQL to query actual data from Wikidata.
	\item In chapter 5, we provide a comparison of the two approaches with respect to their features and show how they evaluate against standard SPARQL queries.
\end{itemize}


\pagebreak

\chapter{Preliminaries}
%\section{Preliminaries}
In this chapter, we introduce some basic concepts of RDF, Wikidata, SPARQL, and GraphQL. Then, we discuss the differences between SPARQL and GraphQL in terms of ease of use by developers in their applications and expressibility.

%\stepcounter{section}
%\setcounter{secnumdepth}{2}
\section{RDF}
%\subsection{RDF}
The World Wide Web consists of data published in various formats such as PDF, CSV and many forms of plain text \cite{Ruth2013}. Linked Data turns the web into a global database where data can be reused and shared across to everybody. Resource Description Framework (RDF) is a framework used to represent information available in the Web \cite{R.Cyganiak2014}. In the context of graphs, RDF is used for describing and exchanging graphs. The graphs specified by RDF are directed edge-labelled graphs. This means that the edges connect source nodes to target nodes, and have labels. It can be the case that there are multiple edges between the same nodes. However, these edges must have different labels. Figure~\ref{fig:figure 1} shows how knowledge about the chemical element Helium can be represented using a directed edge-labelled graph.

\begin{figure}[h]
  \centering
  \includegraphics[width=0.80\linewidth]{images/del_graph.drawio.pdf}
  \caption{Directed edge-labelled graph describing Helium}
  \label{fig:figure 1}
\end{figure}

RDF Schema (RDFS) is the Vocabulary Description Language for RDF. Basically, it defines the vocabulary for RDF data. This means that it describes:

\begin{itemize}
	\item the basic concepts and abstract syntax of RDF such as resources and classes\footnote{https://www.w3.org/TR/rdf-concepts/ .}
	\item the formal semantics of RDF\footnote{https://www.w3.org/TR/2014/REC-rdf11-mt-20140225/ .}
	\item the different concrete syntaxes such as Triples, which is shown in section XYZ
\end{itemize}

According to RDFS, resources are divided into groups known as classes. Each member of a class is called an instance of that class, and is itself also a resource. The relationship between subject and object resources is described via RDF properties Essentially, the predicate of an RDF statement is an instance of RDF property. For instance, to identify that a resources is an instance of a class we use the predicate rdf:type which is in turn an instance of RDF property. W3cC provides a thorough documentation on RDF Schema.\footnote{w3.org/TR/2014/REC-rdf-schema-20140225/ .} In Figure 1 we see that Helium is an instance of the class chemical element. The property instance of describes the relation between the subject Helium and object chemical element.
\anas{Is the concept of property and predicate clear?}

In order to exchange graphs across the web we need to identify the resources uniquely. For this we use IRIs which are basically identifiers in RDF. The graph shown in Figure~\ref{fig:figure 1} can be represented using an RDF graph. Formally, the building blocks of RDF graphs are IRIs, literals and blank nodes. These are defined as follows.

\subsection*{IRI}
%\subsubsection{IRIs}
A Uniform Resource Identifier (\acrshort{URI}) is a sequence of a subset of ASCII characters that identifies any web resource by using a name, a location, or both. They have a scheme, authority, path, and query and fragment, where all parts other than scheme and path are optional. URIs are of the form \textbf{scheme:[//authority]path[?query][\#fragment]}. For example, \textit{http://www.wikidata.org/entity/Q560} is an IRI that identifies the chemical element Helium on Wikidata. A Uniform Resource Locator (\acrshort{URL}) is a subset of URI that is used to specify the location of a digital document.

An Internationalized Resource Identifier (\acrshort{IRI}) is a generalized form of URI that helps to distinguish resources with Unicode. Basically, the character set in URI is extended to the Universal Coded Character Set. This enables it to contain any Latin and non-Latin characters except the reserved characters.

In RDF an IRI is used as a name (can be thought of as an ID) for graph nodes. It defines the resources that appears as nodes or edge labels in a RDF graph. There are already several pre-existing IRIs available for common use. New domain specific IRIs can be created based on the application. However, we must ensure there no conflicts with other IRIs available on the web.

\subsection*{RDF Literals}
%\subsubsection{RDF Literals}

An RDF literal consists of three essential elements: a lexical value, a datatype IRI and an optional language tag. The lexical value is a string\footnote{RDF is based on Unicode strings.} that corresponds to a particular literal value in the value space, where value space is the set of all possible values that a datatype can have. There are many datatypes\footnote{A full list is available on the W3C's section on RDF datatypes: www.w3.org/TR/2014/REC-rdf11-concepts-20140225/\#section-Datatypes.} in RDF some of which are string, Boolean, decimal and integer.

The datatype IRI refers to a datatype that defines which strings are valid (belong in the lexical space), the value space and the lexical-to-value mapping \cite{ Bonduel2019}. This mapping is essentially a function that maps each string from the lexical space to an element in the value space. The \acrshort{W3C} standard XML Schema defines the datatypes and their IRIs. For example, decimals are identified by the IRI http://www.w3.org/2001/XMLSchema\#decimal. W3C has a good documentation on the different XML Schema built-in datatypes \cite{ R.Cyganiak2014}.

The optional language tag helps to provide human-readable labels to RDF literals. A literal is a language-tagged string is of the form "string"@language.\footnote{Here language is a well-formed language tag (after \acrshort{BCP47}).} The datatype IRI\footnote{It is never used in syntax.} of such literals is http://www.w3.org/1999/02/22-rdf-syntax-ns\#langString.

RDF literals are used to represent resources that have values belonging to datatypes. Each literal can have only one datatype. For example, the boiling point of Helium would be a RDF literal represented as \texttt{-268.9^^xsd:decimal} and its chemical formula as \texttt{"He"@en}, which is a language-tagged string. Literals are drawn as rectangular nodes in RDF graphs. 

\subsection*{Blank Nodes}
%\subsubsection{Blank Nodes}
A blank node in RDF, also known as a bnode, does not identify a specific resource as IRIs or literals do. It is used as a placeholder for some node, i.e., it is used to say that something with the given relationship exits at the position without specifying what the node is.


\subsection{RDF Graph}
\begin{definition}[RDF Graph]	
An RDF graph is a directed edge-labelled graph composed of a set of triples. A triple, also known as statement, represents the relationship between a subject and an object, linked by a predicate as shown in Figure~\ref{fig:figure 2}. Formally, each triple consist of the following elements:  

\begin{itemize}
	\item a subject node that is an IRI or a blank node
	\item a predicate edge that is an IRI
	\item an object node that is an IRI, a blank node, or a literal
\end{itemize}	
\end{definition}

\begin{figure}[h]
  \centering
  \includegraphics[width=0.75 \linewidth]{images/rdf_relation.drawio.pdf}
  \caption{An RDF graph with Subject and Object nodes connected via a predicate edge}
  \label{fig:figure 2}
\end{figure}



Figure~\ref{fig:figure 3} shows an RDF graph based on our example represented in Figure~\ref{fig:figure 1}. Our main interest is in querying the knowledge graph Wikidata, and so all the data correspond to the resources in its knowledge base. In Wikidata the subject and object represent items, and the predicate represents properties. All items and properties are identified as Unique IDs. For example, the item \textit{Helium} has the ID of \texttt{Q560}, and the property \textit{chemical formula} has the ID \texttt{P274}. These are not understood by humans and have a label property that makes them understood. Moreover, items belong to the namespace \texttt{http://www.wikidata.org/entity/} (prefixed by \texttt{wd}) and properties to \texttt{http://www.wikidata.org/prop/direct/} (prefixed by \texttt{wdt}). As a result, \textit{Helium} would have the IRI \texttt{http://www.wikidata.org/entity/Q560} (\texttt{wd:Q560}) and \textit{chemical formula} the IRI \texttt{http://www.wikidata.org/prop/direct/P274} (\texttt{wdt:P274}). Section XYZ gives an elaborate understanding of entities and the namespaces they belong to in Wikidata.

From Figure~\ref{fig:figure 3} we understand that \textit{Helium} \texttt{(\gls{Q560})} is an \textit{instance of} \texttt{(P31)} of \textit{chemical element} \texttt{(Q11344)}. It has a human understandable english \textit{label} called \textit{helium} and the \textit{chemical formula} \texttt{(P274)} of \textit{He}. Its \textit{boiling point} \texttt{(P2102)}, \textit{melting point} \texttt{(P2102)} and \textit{density} \texttt{(P2054)} are \textit{-268.9~°C}, \textit{0.1785~°C} and \textit{-272-05~kg/m3} respectively. Helium has a \textit{discoverer/inventor} \texttt{(P61)} by the name of \textit{Pierre Janssen} \texttt{(Q298581)}. His \textit{place of birth} \texttt{(P19)} was \textit{Paris} \texttt{(Q90)} that belongs to the \textit{country} \texttt{(P17)} of \textit{France} \texttt{(Q142)}.

\begin{figure}[h]
  \centering
  \includegraphics[width=0.75 \linewidth]{images/rdf_graph.drawio.pdf}
  \caption{RDF graph describing Helium}
  \label{fig:figure 3}
\end{figure}

%\begin{table}[b!]
%	\begin{center}
%		\caption{Abbreviation/IDs and their meanings.}
%		\label{tab: table 1}
%		\begin{tabular}{c|c}
%%			\textbf{Benchmarking tool} & \textbf{Resource monitoring tool} & \textbf{License} & \textbf{Updated} \\ \hline
%			wd & http://www.wikidata.org/entity/ \\ \hline
%			wdt & http://www.wikidata.org/prop/direct/ \\ \hline
%			rdfs & http://www.w3.org/2000/01/rdf-schema\# \\ \hline
%			Q560 & Helium \\ \hline
%			Q298581	& discoverer/inventor \\ \hline
%			Q90 & Paris \\ \hline
%			P31	& instance of \\ \hline
%			P274 & chemical formula \\ \hline
%			P2102 & boiling point \\ \hline
%			P2101 & melting point \\ \hline
%			P2054 & density \\ \hline
%			P61	& discoverer/inventor \\ \hline
%			P19	 & place of birth \\ \hline
%			P625 & coordinate location
%		\end{tabular}
%	\end{center}
%\end{table}


\subsection{Serialisations}
For exchanging graphs across the web, we need a syntactical representation of RDF. There are different formats available, the most common ones are N-Triples, Turtle, JSON-LD, RDF/XML and RDFa. In this report we focus on N-Triples and Turtle.

\subsubsection{N-Triples}
N-Triples represents RDF graphs in a simple line-based format.\footnote{Full specification available at: https://www.w3.org/TR/n-triples/ .} Every triple is encoded in a single line. The IRIs are written within pointy brackets and literals are written as lexical value\textasciicircum \textasciicircum datatype-IRI. Blank nodes are represented as \_:stringID, where stringID can be any string used to identify the blank node in the document. After every element of a triple there is a whitespace, and all the lines end with a dot. We can use comments using hash symbol after the end of every triple in a line or in a single dedicated line, and they are treated as white spaces. The files are saved with a \textit{.nt} extension.

Listing~\ref{listing:listing1} shows the representation of the RDF graph in Figure~\ref{fig:figure 3} in N-triples format. We have given line breaks for better readability.

\begin{minipage}{\linewidth}
\begin{lstlisting}[columns=fullflexible, label=listing:listing1, caption={RDF graph represented in N-triples syntax}, language=SPARQL]

<http://www.wikidata.org/entity/Q560> <http://www.wikidata.org/prop/direct/P31> 
%\phantom{<http://www.wikidata.org/entity/Q560> <http://www.}%<http://www.wikidata.org/entity/Q11344> .
		                                                
<http://www.wikidata.org/entity/Q560> <http://www.w3.org/2000/01/rdf-schema#label> 
%\phantom{<http://www.wikidata.org/entity/Q560> <http://www.}%"helium"@en .

<http://www.wikidata.org/entity/Q560> <http://www.wikidata.org/prop/direct/P274> 
%\phantom{<http://www.wikidata.org/entity/Q560> <http://www.}%"He"@en .

<http://www.wikidata.org/entity/Q560>  <http://www.wikidata.org/prop/direct/P2102> 
%\phantom{<http://www.wikidata.org/entity/Q560> <http://www.}%"-268.9"^^<http://www.w3.org/2001/XMLSchema#decimal> .

<http://www.wikidata.org/entity/Q560> <http://www.wikidata.org/prop/direct/P2101> 
%\phantom{<http://www.wikidata.org/entity/Q560> <http://www.}%"0.1785"^^<http://www.w3.org/2001/XMLSchema#decimal> .

<http://www.wikidata.org/entity/Q560> <http://www.wikidata.org/prop/direct/P2054> 
%\phantom{<http://www.wikidata.org/entity/Q560> <http://www.}%"-272.05"^^<http://www.w3.org/2001/XMLSchema#decimal> .

<http://www.wikidata.org/entity/Q560> <http://www.wikidata.org/prop/direct/P61> 
%\phantom{<http://www.wikidata.org/entity/Q560> <http://www.}%<http://www.wikidata.org/entity/Q298581> .

<http://www.wikidata.org/entity/Q298581> <http://www.wikidata.org/prop/direct/P19> 
%\phantom{<http://www.wikidata.org/entity/Q560> <http://www.}%<http://www.wikidata.org/entity/Q90> .

<http://www.wikidata.org/entity/Q90> <http://www.wikidata.org/prop/direct/P17> 
%\phantom{<http://www.wikidata.org/entity/Q560> <http://www.}%<http://www.wikidata.org/entity/Q142> . 

\end{lstlisting}
\end{minipage}

\subsubsection{Turtle}
Turtle is an easy to read representation of RDF graphs. It extends the N-Triples format by providing several simplifications.\footnote{Full specification available at: https://www.w3.org/TR/turtle/ .} We can use prefix declarations and base namespaces at the beginning of the file to shorten IRIs. Turtle allows us to avoid repetition. We can use a semicolon at the end of a line instead of a dot if we know the next line has the same subject. Consequently, the next line will only have a predicate and object omitting the subject. Also, we can use a comma at the end of a line if we know the next line will start with the same subject and predicate. Analogously, the next line will only have a an object omitting the subject and the predicate. Blank nodes are represented using only square brackets. Additionally, we can provide predicate-object pairs within the square brackets to give further triples keeping the blank node as the subject. Turtle also provides a shorthand syntax for writing numbers. Numbers of the datatype integer, decimal and double can be written without quotes and datatype-IRIs. Boolean values can be written directly as either "\textit{true}" or "\textit{false}". The files are saved with a \textit{.ttl} extension.

Listing~\ref{listing:listing2} shows the representation of the RDF graph in Figure~\ref{fig:figure 3} in Turtle format. 

\begin{minipage}{\linewidth}
\begin{lstlisting}[label=listing:listing2, caption={RDF graph represented in Turtle syntax}, language=SPARQL]

PREFIX wd: <http://www.wikidata.org/entity/>
PREFIX wdt: <http://www.wikidata.org/prop/direct/>
PREFIX rdfs: <http://www.w3.org/2000/01/rdf-schema#>
PREFIX xsd: <http://www.w3.org/2001/XMLSchema#>
PREFIX geo: <http://www.opengis.net/ont/geosparql#>

wd:Q560 wdt:P31 wd:Q11344 ;
%\phantom{wd:Q560 }% rdfs:label "helium"@en ;
%\phantom{wd:Q560 }% wdt:P274 "He"@en ;
%\phantom{wd:Q560 }% wdt:P2102 -268.9 ;
%\phantom{wd:Q560 }% wdt:P2101 0.1785 ;
%\phantom{wd:Q560 }% wdt:P2054 -272.05 ;
%\phantom{wd:Q560 }% wdt:P61 wd:Q298581 .
wd:Q298581 wdt:P19 wd:Q90 .
wd:Q90 wdt:P17 wd:Q142 .

\end{lstlisting}
\end{minipage}


\section{Wikidata}
%\subsection{Wikidata}
Wikidata is a free and publicly available knowledge base that can be read and edited by both humans and machines. It is one of the many projects by Wikimedia Foundations such as Wikipedia, Wikibooks, Wikimedia Commons and Wikitionary.  Wikidata was created in 2012 at Wikimedia Deutschland by a community of volunteers. These volunteers edit and control all content. As of December 2022, Wikidata has more 23000 active editors.\footnote{https://wikidata.wikiscan.org/ .} Wikidata provides a website where data can be viewed and also edited \cite{Foundationa}.

One of the original purposed behind the creation of Wikidata was to help its sister projects. Initially, Wikipedia and its sister projects used to maintain their own lists of interlanguage links. This refers to links between Wikipedia articles about the same topic but in different languages. After 2012, these interlanguage links were provisioned via Wikidata. Wikidata is also used to display data shown inside the pages in Wikipedia. The usage of this mainly depends on the language of the Wiki. For instance, in the case of some languages, parts of the Wikipedia pages are created automatically from the data in Wikidata. Others, especially those of smaller languages that are not widely used, use Wikidata to create placeholder pages when an article may not be in their respective language. In 2018, around 59\% of Wikidata information was used in English Wikipedia articles, although mostly for external identifiers and coordinate locations \cite{Wikipedia2017}. 

Wikidata stores information in the form of structured data in a database \cite{Tharani2021}. This is not the case for its sister projects as they contain unstructured data stored. The information on their web pages is not directly given a structure in the form of tables or lists. Wikidata acts as a central storage for these projects and focuses on providing a structure for their data \cite{Wikidata2014}. Additionally, Wikidata also supports linked data. This means that the data stored can be linked to datasets and databases like Google Books and OmegaWiki. Wikidata can also be used for quality checks against Wikipedia articles. This is useful when information about a specific topic needs to be known and the solution is easily found by querying the knowledge graph, in this case Wikidata.

Moreover, there is also a wide range of commercial and research oriented applications for Wikidata. This is due to the fact that it has a large amount of real world data. For instance, Wikidata has external usages in many large organizations such as Eurowings, Google, Apple and Amazon. This includes tasks such as data integration, authority control, identity providing and data-driven journalism. In the field of research, Wikidata is used for collecting test data for knowledge graph related algorithms and training data for machine learning projects.

Wikidata is built on the RDF framework. However, it does not define its resources in terms of RDF. Instead, it has its own model known as Wikibase data model.\footnote{https://www.mediawiki.org/wiki/Wikibase/DataModel.} This distinction creates an abstract layer between its model and RDF. Consequently, there are similarities with RDF’s W3C standards but there are also some important differences. For instance, the Wikidata property \textit{instance of} (\texttt{P31}) is semantically equivalent to the property \textit{rdf:type} in RDF. The value of \texttt{P31} is a class that is itself an item. Wikidata offers an explanation for some of the standards properties in Wikidata that correspond to the ones in RDF \cite{Foundation}. Statements (triples) in Wikidata can have annotations (qualifiers) and references.

The basic elements in Wikidata are entities, also known as resources. These are mainly items and properties. The next section describes entities in Wikidata. 

\subsection{Entities}
Wikidata maintains its data structure by pages. Each entity has a dedicated page for itself. Basically, every element on which Wikidata has structured data is known as an entity\cite{Erxleben2014}. Entities are identified by an unique ID and not by names or labels. Currently, Wikidata has mainly three types of entities – item, property and lexeme. Other extensions can define new entity types such as MediaInfo and subentities like Form and Sense. We will discuss about items and properties. The rest are out of the scope of this report.

Items are real-world objects, concepts or events. According to RDF terminology, items are instances and classes. Instead of a human understandable name, they are identified by a QID - an ID prefixed with the letter "Q" and followed a number. Items in Wikidata belong to the main namespace - \textit{http://www.wikidata.org/wiki/QID}. Every item constitutes of the following main parts - labels, descriptions, aliases, sitelinks and statements\cite{Erxleben2014}.

Labels, descriptions and aliases are multilingual which help to find the respective item. Since items are identified by an ID, these are used to identify the items clearly. Sitelinks provide links about an individual item in Wikidata to external pages on other Wikimedia sites such as Wikipedia and its other sister projects. The most important part are statements. 

A statement in Wikidata consists of a claim, references and a rank. Claims are property-value pairs, which along with the item form a RDF triple, where the item is the subject. Claims can also contain optional qualifiers that provide some additional information for the claim. This information is a property-value pair that refers to the main part of the statement instead of the item itself\cite{Erxleben2014}. References point to resources that support the claim. An item can have several statements for the same property, and all of them might not necessarily be important or relevant. Ranks are used as a quality factor to distinguish between several statements. A Wikidata statement can have one of three types of ranks - "normal", "preferred" and "deprecated". By default, a statement has the normal rank unless changed to preferred or deprecated. A statement with preferred rank means it should be given priority over the normal ranked statements. A deprecated rank indicates that the statement is incorrect or under discussion, and it may have a reference. Deprecated statements are kept either for the sake for completion or to prevent users from constantly adding or removing them. Wikidata has around 100 million items\footnote{https://grafana.wikimedia.org/d/000000167/wikidata-datamodel.} and around 1.43 billion item statements as of December 2022.\footnote{https://grafana.wikimedia.org/d/000000175/wikidata-datamodel-statements.}

Figure~\ref{fig:figure 4} shows an excerpt of the Wikidata page on the item Helium.\footnote{https://www.wikidata.org/wiki/Q560.} We can see Helium has the QID of Q560, and has labels, descriptions and aliases in different languages. It has a sitelink for the item in Wikipedia offered in 186 languages. Helium has a statement that indicates that Helium has the instance of property with chemical element as the values. The property-value pairs follows, hydrogen and followed by, lithium are qualifiers. This statement hence gives us the information that Helium is an instance of chemical element, and it comes after the element hydrogen and if followed by the element lithium. This statement has no references and has the normal rank (indicated by the middle portion greyed).

\begin{figure}[h]
  \centering
  \includegraphics[width=0.75 \linewidth]{images/helium.pdf}
  \caption{An excerpt of the page on \textit{Helium} in Wikidata}
  \label{fig:figure 4}
\end{figure}

Properties in Wikidata resemble RDF properties and are essentially attributes for describing entities. They are identified by a PID - an ID prefixed with the letter "P" and followed a number. They belong to the property namespace in wikidata - http://www.wikidata.org/wiki/Property:PID. Like items, they also have labels, descriptions, aliases and statements but no sitelinks. However, they has an additional part called datatype that determines which values they accept, such as string, quantity and time.\footnote{Wikidata provides a list of all the datatypes: https://www.wikidata.org/wiki/Special:ListDatatypes.}

Figure~\ref{fig:figure 5} shows an excerpt of the property instance of page on Wikidata.\footnote{https://www.wikidata.org/wiki/P31.} The property has a PID of P31 along and with multilingual labels, descriptions and aliases. From the statement we understand that it this property also has an instance of property with Wikidata property being the value. The statement has no qualifiers or references, and has the normal rank. The instance of property accepts the datatype item as a value.

\begin{figure}[h]
  \centering
  \includegraphics[width=0.75 \linewidth]{images/instance_of.pdf}
  \caption{An excerpt of the page on \textit{instance of} in Wikidata}
  \label{fig:figure 5}
\end{figure}

\subsection{Querying Wikidata}

The easiest and most popular way to query Wikidata is through the Wikidata Query Service (WDQS). This is Wikidata's SPARQL endpoint. We can use this service two ways. Firstly, we can write queries in SPARQL directly on the web user interface of the service\footnote{https://query.wikidata.org/ .} and obtain the results in different formats like table, tree, graph, etc. Secondly, the service can also be used progmatically by submitting GET or POST requests.\footnote{https://query.wikidata.org/sparql.}

Another popular way to query Wikidata is by using the Wikidata API.\footnote{https://www.wikidata.org/wiki/Special:ApiSandbox.} However, this API should mainly be used when we want to edit the contents of Wikidata or get data about entities like revision history.

Wikidata dumps is useful when we know our result set will be significantly large or if we want to set up our own local query service. These dumps are full exports of all the available entities in Wikidata.\footnote{https://dumps.wikimedia.org/ .} To get started you should download the latest complete dump.\footnote{https://dumps.wikimedia.org/wikidatawiki/latest/ .} Wikidata also mentions some other ways to accessing Wikidata's data like Search and Linked Data Fragments endpoint, the complete list and usage of which can be found on Wikidata's Data Access webpage \cite{ Wikidata2022}.


\section{SPARQL}
%\subsection{SPARQL}
SPARQL Protocol and RDF Query Language (SPARQL)\footnote{https://www.w3.org/TR/sparql11-overview/ .} is a W3C recommended query language for RDF. This means it allows to query any data source that can be mapped to RDF. It is also a HTTP-based protocol for linked open data on the web. This enables the transmission of SPARQL queries and results between a client and a SPARQL engine. The first working draft for SPARQL was released in 2004 and it became a W3C Recommendation in 2008 \cite{Perez2009}. 

Queries in SPARQL are based on matching graph patterns and can be used to retrieve, add or delete data in the RDF based dataset. In section XYZ we saw that RDF data is based on triples - subject, object and predicate. Consequently, a query in SPARQL consists of a set of triple patterns. Each of the elements of the triple can be a variable (a string beginning with ? or \$) that needs to be queried. The solution to the variables is obtained by matching the query patterns to the triples in the dataset.

There are four forms of queries - SELECT, ASK, CONSTRUCT and DESCRIBE. 
\begin{itemize}
\item SELECT queries select some or all the pattern matches and provides the results in a tabular format
\item ASK queries check whether there is at least one match and the result is true or false
\item CONSTRUCT queries create an RDF graph based on the query results
\item DESCRIBE queries return a RDF graph providing additional information on each results
\end{itemize}

In our work, we only consider SELECT queries. These consist of the following major blocks:
\begin{itemize}
\item Prologue: PREFIX and BASE keywords that function similarly to those in RDF turtle format
\item Select clause: SELECT keyword followed by either a list of variables and variable assignments, or by *
\item Where clause: WHERE keyword followed by a query graph pattern to be matched
\item Solution set modifiers: Change the set of solutions using modifiers such as LIMIT and OFFSET
\end{itemize}

The select and where clauses are mandatory, the rest being optional. Other optional features are filters, groups, query operators such as UNION, OPTIONAL and BIND, and aggregates. A full specification for the query language can be found on the official W3C documentation\cite{Seaborn}.

Listing~\ref{listing:listing3} shows an example of querying Wikidata using SPARQL. We want to get a a list of all chemical elements, along with their English labels, that have a chemical formula, boiling point, melting point, density, an inventor/discoverer, birth place of that inventor/discoverer and the country that the place belongs to. Since there might be several results, we are limiting them to five using the LIMIT keyword. The namespace http://www.wikidata.org/entity/ is used for items when querying. We are interested in the truthy values of the properties and so the namespace http://www.wikidata.org/prop/direct/ is used for properties. Truthy values are essentially the values for which the statement has the best non-deprecated rank. This means that if a statement has preferred rank then that statement is considered to be truthy. Otherwise, the normal ranked statement is taken to be truthy. The PREFIX is optional since Wikidata recognizes the short forms wd and wdt automatically. Turtle syntax that we saw in section XYZ can be applied in SPARQL. The query can be run in Wikidata's query service.\footnote{https://query.wikidata.org/ .}  

\begin{minipage}{\linewidth}
\begin{lstlisting}[label=listing:listing3, caption={Querying Wikidata with SPARQL}, language=SPARQL]

PREFIX wd: <http://www.wikidata.org/entity/>
PREFIX wdt: <http://www.wikidata.org/prop/direct/>
SELECT *
WHERE {
  ?element wdt:P31 wd:Q11344 ;
  %\phantom{?element }% wdt:P274 ?element_formula ; 
  %\phantom{?element }% wdt:P2102 ?boiling_point ;
  %\phantom{?element }% wdt:P2101 ?melting_point ;
  %\phantom{?element }% wdt:P2054 ?density ;
  %\phantom{?element }% wdt:P61 ?discoverer .
  ?discoverer wdt:P19 ?place_birth .
  ?place_birth wdt:P17 ?country .
  FILTER(LANG(?element_label)="en")
}LIMIT 5

\end{lstlisting}
\end{minipage}

Table 1 shows the results obtained in a tabular form. Among the results, there is the element Helium (Q560) that we have used as an example in Fig 1 and Fig 3. 

\begin{table}[h]
	\begin{center}
		\caption{Results of the SPARQL query in Listing 2}
		\label{tab: table 1}
		\begin{tabular}{ccccccccc}
		
%		\textbf{element} & \textbf{element_formula} & \textbf{element_label} & \textbf{boiling_point} & \textbf{melting_point} & \textbf{density} & \textbf{discoverer} & \textbf{place_birth} & \textbf{country} \\ \hline
			\toprule
			
			\textbf{element} & \textbf{element\textunderscore formula} & \textbf{element\textunderscore label} & \textbf{boiling\textunderscore point} & \textbf{melting\textunderscore point} & \textbf{density} & \textbf{discoverer} & \textbf{place\textunderscore birth} & \textbf{country} \\ 
		
			\midrule
			
			wd:Q560 & He & helium	& -268.9 & -272.05 & 0.1785 & wd:Q298581 & wd:Q90 & wd:Q142 \\
			
			wd:Q560 & He & helium & -268.9 & -272.05 & 0.1785 & wd:Q950726 & wd:Q4093 & wd:Q145 \\ 
			
			wd:Q560 & He & helium & -268.9 & -272.05 & 0.1785	 & wd:Q127959 & wd:Q623765 & wd:Q145 \\ 
			
			wd:Q670 & Si & silicon & 4271 & 2570	& 2.329	& wd:Q151911 & wd:Q1451001 & wd:Q34 \\ 
			
			wd:Q568	& Li & lithium	& 1317	& 180.5	& 0.535	& wd:Q313568 & wd:Q10495519 & wd:Q34 \\
			
			\bottomrule

		\end{tabular}
	\end{center}
\end{table}

\section{GraphQL}

GraphQL (Graph Query Language) is an open source query language for APIs (Application Programming Interfaces) and a runtime for executing those queries against existing data. It describes data structured in a graph format – a collection of objects (nodes) connected to each other by some kind of relationships (edges). Runtime is usually implemented by a server.

GraphQL is usually served over HTTP through a GraphQL server. A GraphQl server consists of two main parts – schema and resolver. The API developers create a schema that is strictly typed and describes all the possible data a client can query using the service. The schema specifies object types and fields along with operations on those types. The object type represents the kind of object that can be requested by a client. 

There are three operations types – query, mutation and subscription.  We look at the query operation in this chapter; the other two are outside the scope of this report. Queries are used to fetch or read data. When compared to REST (Representational State Transfer),  queries operations in GraphQL are similar to GET requests. A resolver in GraphQL server is a function that is associated with every field and contains instructions on how to process that particular field. In other words, the resolver is responsible for retrieving a value from the data source. 

GraphQL is data agnostic, i.e., it is not concerned where the data source is located. The data could be stored in any source such as a database or a micro-service as shown in Figure XYZ. With a single API call, GraphQL can aggregate data from multiple sources and resolve the data to the client. This is one of the advantages GraphQL has over REST API where the latter would require several HTTP calls to access data from multiple sources. Apart from being data agnostic, GraphQL is also language agnostic. This means that GraphQL services, such as the schema and resolvers, can be written in any programming language such as JavaScript or Python. 

For the sake of human readability, GraphQL specification has its own Schema Definition Language (SDL). It is simple to write and understand schemas in SDL, and is similar to the language that we use to write queries. Listing XYZ shows a schema written in SDL. This schema can be in the GraphQL server against which clients can send queries for instances of chemical elements that have a name, chemical formula and boiling point. The exclamation mark means that the corresponding field is non-nullable and it is expected that GraphQL will give a value when the field is queried. A complete guide on schemas and types can be found on the official documentation from GraphQL.\footnote{https://graphql.org/learn/schema/ .} 

\begin{minipage}{\linewidth}
\begin{lstlisting}{gql.py:GraphqlLexer -x}[label=listing:listing4, caption={Schema used to query a chemical element}]
type Query{
	chemicalElement: ChemicalElement
}

type ChemicalElement{
	name: String!
	chemicalFormula: String!
	boilingPoint: Float!
}

\end{lstlisting}
\end{minipage}

Listing XYZ shows a query that can be used against this schema and the results that could be obtained. The results obtained are in JSON format. The official GraphQL website provides a comprehensive documentation on querying a GraphQL server. \footnote{ https://graphql.org/learn/queries/ .} 

\begin{minipage}{\linewidth}
\begin{lstlisting}[label=listing:listing5, caption={Query to fetch chemical elements and their properties}]
query QueryChemicalElement (limit 2){
    chemicalElement{
		name
		chemicalFormula
		boilingPoint
	}
}
\end{lstlisting}
\end{minipage}

\begin{minipage}{\linewidth}
\begin{lstlisting}[label=listing:listing5, caption={Query to fetch chemical elements and their properties}]
{
	"data":{
		"chemicalElement":{
			"name": "helium",
			"chemicalFormula": "He",
			"boilingPoint": -268.9
		},
		{	
			"name": "silicon",
			"chemicalFormula": "Si",
			"boilingPoint": 4271
		}
	}
}

\end{lstlisting}
\end{minipage}

\section{GraphQL vs SPARQL}

GraphQL and SPARQL are query languages developed with different goals in mind. GraphQL was designed mainly to wrap REST APIs in a graph like shape. This would allow fetching related data in a single request with the aid of schemas. In REST API, the same would require calling multiple endpoints. SPARQL, on the other hand, was developed mainly as a query language for RDF graphs.

SPARQL is immensely popular in the field of research and academia. However, it has not seen much growth in commercial applications. GraphQL, on the other hand, is more popular among software and web developers. There are some valid reasons for this.  

Firstly, many developers are still not familiar with the Linked Data model of RDF and SPARQL. They are more used to working with technologies such as GraphQL and RESTAPIs. Secondly, GraphQL is simple to learn and work with since it has human-oriented syntax. Among other things, this benefits application development. SPARQL however, proves to be more complex. It has more syntactic verbosity owing to the descriptive nature of writing queries. The produced output from SPARQL queries contains a lot of unnecessary metadata that is not useful for web developers and need to be further parsed for using in web applications \cite{Lisena2018}. 

Moreover, retrieval of data from knowledge graphs using SPARQL is time consuming and proves to be a steep learning curve, the reason being that there is limited documentation available for proper ontology descriptions and examples of using queries for SPARQL endpoints \cite{Angele2022}. One of the other reasons for using GraphQL over SPARQL is that developers are more equipped in using nested objects that GraphQL offers \cite{Taelman2018}. They lack the experience to work with triples which is the main foundation of SPARQL. Moreover, fewer supporting tools like libraries and frameworks exist for working and developing with SPARQL than with GraphQL \cite{Taelman2018}. 

On the other hand, SPARQL also has some advantages over GraphQL. SPARQL queries represent full graphs while those of GraphQL represent trees \cite{Taelman2018}. This make SPARQL more expressive. RDF provides a system to build detailed structures from the meaning of data, and is hence more complete and capable than GraphQL schemas \cite{Dresslar2019}. Since SPARQL works with the schema organization of RDF, this makes it more powerful than GraphQL. With SPARQL we can write complex queries that can be used to retrieve or modify data \cite{Angele2022}. As a result it is capable to satisfy many complex use cases.

Also since GraphQL alone has no notion of semantics \cite{Taelman2018}, a schema needs to be defined by the GraphQL API developers for every interface they want the client to query. This makes it difficult to integrate data returned when querying multiple different sources. SPARQL however supports federated queries which makes it more powerful and rich. Lastly, when using GraphQL we cannot uniquely identify resources on the web but which is possible using URIs with SPARQL. In other words, GraphQL has no notion of global identifiers \cite{Taelman2018}.

\textcolor{red}{Talk also about differences in terms of complexity and expressivity}


\pagebreak

\chapter{Approaches for Querying RDF Graphs with GraphQL}

In this chapter, we focus on ways of queries RDF Graphs using GraphQL. In section XYZ of the previous chapter, we highlighted the differences between GraphQL and SPARQL. This gives rise to the idea of using GraphQL to query RDF graphs, thereby overcoming the limitations of using SPARQL directly. 
In recent years, there have been attempts at providing approaches to querying linked data represented by RDF via GraphQL. This gave rise to several commercial and open-source solutions. Most notable ones include:

\begin{itemize}
	\item Stardog
	\item TopBraid EDG
	\item Ontotext Platform
	\item GraphQL-LD
	\item HyperGraphQL
	\item UltraGraphQL

\end{itemize}

Stardog\footnote{https://www.stardog.com/} is a commercial solution that offers a graph database called the "Enterprise Knowledge Graph platform"\cite{Angele2022}. The initial versions only allowed querying their stored data using SPARQL. The support for querying using  GraphQL was added since the release of version 5.1. (Work in progress)


\pagebreak

\chapter{Implementing the approaches on Wikidata}

In this chapter we discuss and demonstrate the implementation of querying the knowledge graph Wikidata using both GraphQL-LD and HyperGraphQL. This is followed by the technical details and the setup of both the tools. Then, we conclude the chapter by discussing the differences between both the approaches and their limitations respectively.

\section{GraphQL-LD on Wikidata}

Wikidata offers a SPARQL endpoint https://query.wikidata.org/sparql against which requests can be sent to query its data. In GraphQL-LD, we require a GraphQL query and a JSON-LD context as input files. To demonstrate the implementation we use a practical example where we use the query and JSON-LD context shown in Listings XYZ and XYZ respectively. 

Basically, we want to fetch all the chemical elements from Wikidata and some of their properties. These properties are the chemical formula, boiling point, melting point and density of the elements, along with the person who discovered or invented the elements, and that person’s place of birth and country to which the place belongs. This example was used in Listing 2 where we demonstrated the querying of Wikidata using SPARQL. The only difference is that we do not query for the labels of the chemical elements.

GraphQL-LD does not support the feature of filtering the labels in a specific language. Querying for them would fetch the label for each element in all the available languages in Wikidata for that element. This would unnecessary populate the results without much benefit. Hence, we remove the label field.

\begin{minipage}{\linewidth}
\begin{lstlisting}[label=listing:listing17, caption={Query}]
query {
    checimcalElement @single(scope: all) {
        id 
        chemicalFormula
        boilingPoint
        meltingPoint
        density
        discoverer {
            id
            placeBirth {
               id
                country {
					id
				}
            }
        }
    }
}
\end{lstlisting}
\end{minipage}

\begin{minipage}{\linewidth}
\begin{lstlisting}[label=listing:listing18, caption={JSON-LD Context}]
{
  "@context": {
    "wd": "http://www.wikidata.org/entity/",
    "wdt":"http://www.wikidata.org/prop/direct/",
    "instance": "wdt:P31",
    "chemicalFormula": "wdt:P274",
    "boilingPoint": "wdt:P2102",
    "meltingPoint": "wdt:P2101",
    "density": "wdt:P2054",
    "discoverer": "wdt:P61",
    "placeBirth": "wdt:P19",
    "country": "wdt:P17"
  }
}
\end{lstlisting}
\end{minipage}

The generated SPARQL query is shown in Listing XYZ. 

\begin{minipage}{\linewidth}
\begin{lstlisting}[label=listing:listing19, caption={Generated SPARQL Query}]
SELECT ?checimcalElement_id ?checimcalElement_chemicalFormula ?checimcalElement_boilingPoint ?checimcalElement_meltingPoint ?checimcalElement_density ?checimcalElement_discoverer_id 
?checimcalElement_discoverer_placeBirth_id ?checimcalElement_discoverer_placeBirth_country WHERE {
  ?df_3_0 undefined:checimcalElement ?checimcalElement_id.
  ?checimcalElement_id <http://www.wikidata.org/prop/direct/P274> ?checimcalElement_chemicalFormula;
    <http://www.wikidata.org/prop/direct/P2102> ?checimcalElement_boilingPoint;
    <http://www.wikidata.org/prop/direct/P2101> ?checimcalElement_meltingPoint;
    <http://www.wikidata.org/prop/direct/P2054> ?checimcalElement_density;
    <http://www.wikidata.org/prop/direct/P61> ?checimcalElement_discoverer_id.
  ?checimcalElement_discoverer_id <http://www.wikidata.org/prop/direct/P19> ?checimcalElement_discoverer_placeBirth_id.
  ?checimcalElement_discoverer_placeBirth_id <http://www.wikidata.org/prop/direct/P17> ?checimcalElement_discoverer_placeBirth_country.
}
\end{lstlisting}
\end{minipage}

The above SPARQL query does not produce any results in Wikidata. As we mentioned in chapter XYZ, GraphQL-LD is predicate-oriented. It queries the relationships between nodes. This creates a problem when we want to say that some subject is an instance or type of some object. GraphQL-LD does not recognize the root node to be an instance or type of a node. Instead it treats the root node to be a property. Hence, some workaround needs to be done to achieve the desired results.

We propose three solutions that can be potentially use as a workaround. 

\subsection{Inline-ID based Solution 1}

This solution is inspired by the "Setting an inline id"\footnote{https://github.com/rubensworks/graphql-to-sparql.js\#setting-an-inline-id.} section provided in the documentation of GraphQL-LD. Although it is intended to be used for defining or looking up the id of entities as per the documentation, we realized this option come help overcome the issue we were having when defining an item to be an instance of another item. Listings XYZ and XYZ show the GraphQL query and the corresponding generated SPARQL query generated using the JSON-LD context in Listing XYZ respectively. 

To query against Wikidata, we can provide the available SPARQL endpoint – https://query.wikidata.org/sparql. This generates the results in JSON shown in Listing XYZ. We show only the first three results. The LIMIT solution modifier could have been implemented for this by using "first: 3" in our GraphQL queries. However, GraphQL-LD would have generated a nested SPARQL query, and we decided that the comparison between the proposed solutions would be more prominent and easier to understand using a simpler SPARQL query.

\begin{minipage}{\linewidth}
\begin{lstlisting}[label=listing:listing20, caption={Query}]
query {
    instance(_: chemicalElement) @single(scope: all)
        id
        chemicalFormula
        boilingPoint
        meltingPoint
        density
        discoverer {
            id
            placeBirth {
                id
                country
            }
        }
}
\end{lstlisting}
\end{minipage}

\begin{minipage}{\linewidth}
\begin{lstlisting}[label=listing:listing21, caption={Generated SPARQL Query}]
SELECT ?id ?chemicalFormula ?boilingPoint ?meltingPoint ?density ?discoverer_id ?discoverer_placeBirth_id ?discoverer_placeBirth_country WHERE {
  ?id <http://www.wikidata.org/prop/direct/P31> <http://www.wikidata.org/entity/Q11344>;
    <http://www.wikidata.org/prop/direct/P274> ?chemicalFormula;
    <http://www.wikidata.org/prop/direct/P2102> ?boilingPoint;
    <http://www.wikidata.org/prop/direct/P2101> ?meltingPoint;
    <http://www.wikidata.org/prop/direct/P2054> ?density;
    <http://www.wikidata.org/prop/direct/P61> ?discoverer_id.
  ?discoverer_id <http://www.wikidata.org/prop/direct/P19> ?discoverer_placeBirth_id.
  ?discoverer_placeBirth_id <http://www.wikidata.org/prop/direct/P17> ?discoverer_placeBirth_country.
}
\end{lstlisting}
\end{minipage}

\begin{minipage}{\linewidth}
\begin{lstlisting}[label=listing:listing22, caption={Output}]
[
  {
    "id": "http://www.wikidata.org/entity/Q560",        
    "boilingPoint": -268.9,
    "density": 0.1785,
    "meltingPoint": -272.05,
    "discoverer": {
      "id": "http://www.wikidata.org/entity/Q298581",   
      "placeBirth": {
        "id": "http://www.wikidata.org/entity/Q90",     
        "country": "http://www.wikidata.org/entity/Q142"
      }
    },
    "chemicalFormula": "He"
  },
  {
    "id": "http://www.wikidata.org/entity/Q1119",       
    "boilingPoint": 8316,
    "density": 13,
    "meltingPoint": 4041,
    "discoverer": {
      "id": "http://www.wikidata.org/entity/Q775969",
      "placeBirth": {
        "id": "http://www.wikidata.org/entity/Q727",
        "country": "http://www.wikidata.org/entity/Q55"
      }
    },
    "chemicalFormula": "Hf"
  },
  {
    "id": "http://www.wikidata.org/entity/Q1094",
    "boilingPoint": 3767,
    "density": 7.31,
    "meltingPoint": 314,
    "discoverer": {
      "id": "http://www.wikidata.org/entity/Q77308",
      "placeBirth": {
        "id": "http://www.wikidata.org/entity/Q1731",
        "country": "http://www.wikidata.org/entity/Q183"
      }
    },
    "chemicalFormula": "In"
  },
  ...
]
\end{lstlisting}
\end{minipage}

\subsection{ID based Solution 2}

This solution is proposed after doing some experiments with the source code. The documentation from GraphQL-LD does not provide any reference to this but we consider this to be a potential solution. Initially, when no id was provided as a field in the GraphQL query, the fetched result only contained ids of the queried item, and ignored the rest of the properties. We had to update the source code to overcome this issue.

Listings XYZ, XYZ and XYZ show the GraphQL query, generated SPARQL query and the corresponding first three JSON results after querying Wikidata. The only issue is that the generated SPARQL query includes the id in the SELECT clause even if no id is provided in the SPARQL query, although this does not affect the results and we get them as expected. We can try to correct this issue in future works by updating the source code further.

\begin{minipage}{\linewidth}
\begin{lstlisting}[label=listing:listing23, caption={Query}]
query {
    id (instance: chemicalElement) @single(scope: all) {
        id
        chemicalFormula
        boilingPoint
        meltingPoint
        density
        discoverer {
            placeBirth {
                id
                country
            }
        }
    }
}
\end{lstlisting}
\end{minipage}

\begin{minipage}{\linewidth}
\begin{lstlisting}[label=listing:listing24, caption={Generated SPARQL Query}]
SELECT ?id ?id_id ?id_chemicalFormula ?id_boilingPoint ?id_meltingPoint ?id_density ?id_discoverer_placeBirth_id ?id_discoverer_placeBirth_country WHERE {
  ?id_id <http://www.wikidata.org/prop/direct/P31> <http://www.wikidata.org/entity/Q11344>;
    <http://www.wikidata.org/prop/direct/P274> ?id_chemicalFormula;
    <http://www.wikidata.org/prop/direct/P2102> ?id_boilingPoint;
    <http://www.wikidata.org/prop/direct/P2101> ?id_meltingPoint;
    <http://www.wikidata.org/prop/direct/P2054> ?id_density;
    <http://www.wikidata.org/prop/direct/P61> ?id_discoverer.
  ?id_discoverer <http://www.wikidata.org/prop/direct/P19> ?id_discoverer_placeBirth_id.
  ?id_discoverer_placeBirth_id <http://www.wikidata.org/prop/direct/P17> ?id_discoverer_placeBirth_country.
}
\end{lstlisting}
\end{minipage}

\begin{minipage}{\linewidth}
\begin{lstlisting}[label=listing:listing25, caption={Output}]
[
  {
    "id": {
      "id": "http://www.wikidata.org/entity/Q560",        
      "boilingPoint": -268.9,
      "density": 0.1785,
      "meltingPoint": -272.05,
      "chemicalFormula": "He",
      "discoverer": {
        "placeBirth": {
          "id": "http://www.wikidata.org/entity/Q90",     
          "country": "http://www.wikidata.org/entity/Q142"
        }
      }
    }
  },
  {
    "id": {
      "id": "http://www.wikidata.org/entity/Q1119",
      "boilingPoint": 8316,
      "density": 13,
      "meltingPoint": 4041,
      "chemicalFormula": "Hf",
      "discoverer": {
        "placeBirth": {
          "id": "http://www.wikidata.org/entity/Q727",
          "country": "http://www.wikidata.org/entity/Q55"
        }
      }
    }
  },
  {
    "id": {
      "id": "http://www.wikidata.org/entity/Q1094",
      "boilingPoint": 3767,
      "density": 7.31,
      "meltingPoint": 314,
      "chemicalFormula": "In",
      "discoverer": {
        "placeBirth": {
          "id": "http://www.wikidata.org/entity/Q1731",
          "country": "http://www.wikidata.org/entity/Q183"
        }
      }
    }
  },
  ...
]
\end{lstlisting}
\end{minipage}

\subsection{Fragment based Solution 3}

This last solution is inspired by the inline fragment\footnote{https://github.com/rubensworks/graphql-to-sparql.js\#inline-fragments.} usage of GraphQL-LD. Initially, GraphQL-LD translated the "id ... on chemicalElement" part as "?id rdf:type <IRI of chemicalElement>".\footnote{For our query this would be "?id <http://www.w3.org/1999/02/22-rdf-syntax-ns\#type> <http://www.wikidata.org/entity/Q11344>"} This triple pattern does not correspond to the data model in Wikidata as it uses the "instance of" property instead of "rdf:type" when describing that an item is a type of a class. To overcome this issue we had to modify the source code so that it used "http://www.wikidata.org/prop/direct/P31" instead of "<http://www.w3.org/1999/02/22-rdf-syntax-ns\#type>" when translating GraphQL queries to SPARQL queries when fragments are used. Unlike the other two proposed solutions where the "instance" property must be used in the queries, this solution does not require it. Consequently, no entries for the mapping of "instance" of to its IRI, http://www.wikidata.org/entity/Q11344), is needed in the JSON-LD context.

Listings XYZ, XYZ and XYZ show the GraphQL query, generated SPARQL query and the corresponding first three JSON results fetched from Wikidata. This solution treats the fragments as OPTIONAL patterns in SPARQL. Since it is based on fragmentation, the id is always fetched.

\begin{minipage}{\linewidth}
\begin{lstlisting}[label=listing:listing26, caption={Query}]
query {
    id
    ... on chemicalElement @single(scope: all) {
        id
        chemicalFormula
        boilingPoint
        meltingPoint
        density
        discoverer {
            placeBirth {
                id
                country
            }
        }
    }
}
\end{lstlisting}
\end{minipage}

\begin{minipage}{\linewidth}
\begin{lstlisting}[label=listing:listing27, caption={Generated SPARQL Query}]
SELECT ?id ?id_id ?chemicalFormula ?boilingPoint ?meltingPoint ?density ?discoverer_placeBirth_id ?discoverer_placeBirth_country WHERE {
  OPTIONAL {
    ?id <http://www.wikidata.org/prop/direct/P31> <http://www.wikidata.org/entity/Q11344>;
      <http://www.wikidata.org/prop/direct/P274> ?chemicalFormula;
      <http://www.wikidata.org/prop/direct/P2102> ?boilingPoint;
      <http://www.wikidata.org/prop/direct/P2101> ?meltingPoint;
      <http://www.wikidata.org/prop/direct/P2054> ?density;
      <http://www.wikidata.org/prop/direct/P61> ?discoverer.
    ?discoverer <http://www.wikidata.org/prop/direct/P19> ?discoverer_placeBirth_id.
    ?discoverer_placeBirth_id <http://www.wikidata.org/prop/direct/P17> ?discoverer_placeBirth_country.
  }
}
\end{lstlisting}
\end{minipage}

\begin{minipage}{\linewidth}
\begin{lstlisting}[label=listing:listing28, caption={Output}]
[
  {
    "id": "http://www.wikidata.org/entity/Q560",        
    "boilingPoint": -268.9,
    "density": 0.1785,
    "meltingPoint": -272.05,
    "chemicalFormula": "He",
    "discoverer": {
      "placeBirth": {
        "id": "http://www.wikidata.org/entity/Q90",     
        "country": "http://www.wikidata.org/entity/Q142"
      }
    }
  },
  {
    "id": "http://www.wikidata.org/entity/Q1119",       
    "boilingPoint": 8316,
    "density": 13,
    "meltingPoint": 4041,
    "chemicalFormula": "Hf",
    "discoverer": {
      "placeBirth": {
        "id": "http://www.wikidata.org/entity/Q727",
        "country": "http://www.wikidata.org/entity/Q55"
      }
    }
  },
  {
    "id": "http://www.wikidata.org/entity/Q1094",
    "boilingPoint": 3767,
    "density": 7.31,
    "meltingPoint": 314,
    "chemicalFormula": "In",
    "discoverer": {
      "placeBirth": {
        "id": "http://www.wikidata.org/entity/Q1731",
        "country": "http://www.wikidata.org/entity/Q183"
      }
    }
  },
  ...
]
\end{lstlisting}
\end{minipage}

The above three proposed solution are prospective workarounds for describing the "instance of" relationship between two items in Wikidata. The first two solutions, Inline-ID based and ID based, are more feasible than the Fragment based solution. The way of describing the “instance of” relationship using inline fragments in the third solution  is not intuitive, as they are primarily used to query a field that returns an interface or a union type.\footnote{https://graphql.org/learn/queries/\#inline-fragments.} Between the first two solutions, the second one namely the ID based solution is more intuitive in writing GraphQL queries than the first solution.

\subsection{Default JSON-LD context}

Working with GraphQL-LD requires creating a JSON-LD context that requires some significant effort to create. The IRIs of the items used in the GraphQL query need to be looked up and then inserted into the context. To overcome this constraint we created a default JSON-LD context. This contains a list of all items and properties along with their IRIs available in Wikidata as of 12.10.2022. Also, the source code of GraphQL-LD was modified such that the tool takes the default context as input when no context is provided by the user. This default context was stored in a local directory.

For the creation of such a default context, we needed to extract all the items and properties in Wikidata along with their labels to identify them. For the items, we downloaded the file consisting of all truthy statements\footnote{https://www.mediawiki.org/wiki/Wikibase/Indexing/RDF\_Dump\_Format\#Truthy\_statements.} from the RDF dumps provided by Wikidata.\footnote{https://dumps.wikimedia.org/wikidatawiki/entities/ .} We extracted all the items with their labels from this file and removed duplicates. Since many items can have the same labels, there would be ambiguity when the query involved one of these items. Hence, we remove all instances of items where the duplicates occurred. For the properties, we used a SPARQL query to fetch all the properties in Wikidata along with their labels. Finally, a JSON-LD context was created consisting of all the items, properties and their labels. In total there are 84,535,915 items and 10,317 properties in our default context. For the entire process, we used Python and shell scripts.

\section{HyperGraphQL on Wikidata}

To query RDF data using HyperGraphQL we need to set up a HyperGraphQL instance. This requires a configuration file and an annotated schema. We want to query Wikidata for the same data as we did in Listing XYZ when using GraphQL-LD, except now we can query for labels too as HyperGraphQL support it. Listings XYZ, XYZ and XYZ show the GraphQL query, configuration file and schema. We use the same SPARQL endpoint for Wikidata as we did in GraphQl-LD.

\begin{minipage}{\linewidth}
\begin{lstlisting}[label=listing:listing29, caption={Query}]
{
  ChemicalElement_GET {
    _id
    label(lang: "en")
    chemicalFormula
    boilingPoint
    meltingPoint
    density
    discoverer {
      _id
      label(lang: "en")
      placeBirth {
        _id
        label(lang: "en")
        country {
          _id
          label(lang: "en")
        }
      }
    }
  }
}
\end{lstlisting}
\end{minipage}

\begin{minipage}{\linewidth}
\begin{lstlisting}[label=listing:listing30, caption={Configuration}]
{
    "name": "wikidata-hgql",
    "schema": "schema/schema_wikidata.graphql",
    "server": {
        "port": 8081,
        "graphql": "/graphql",
        "graphiql": "/graphiql"
    },
    "services": [
        {
            "id": "wikidata-sparql",
            "type": "SPARQLEndpointService",
            "url": "https://query.wikidata.org/sparql",
            "graph": "",
            "user": "",
            "password": ""
        }
    ]
}
\end{lstlisting}
\end{minipage}

\begin{minipage}{\linewidth}
\begin{lstlisting}[label=listing:listing31, caption={Schema}]
type __Context {
    ChemicalElement:   _@href(iri: "http://www.wikidata.org/entity/Q11344")
    label:  _@href(iri: "http://www.w3.org/2000/01/rdf-schema#label")
    chemicalFormula:  _@href(iri: "http://www.wikidata.org/prop/direct/P274")
    boilingPoint:    _@href(iri: "http://www.wikidata.org/prop/direct/P2102")
    meltingPoint:  _@href(iri: "http://www.wikidata.org/prop/direct/P2101")
    density:  _@href(iri: "http://www.wikidata.org/prop/direct/P2054")
    discoverer:    _@href(iri: "http://www.wikidata.org/prop/direct/P61")
    Human:  _@href(iri: "http://www.wikidata.org/entity/Q5")
    placeBirth:    _@href(iri: "http://www.wikidata.org/prop/direct/P19")
    City: _@href(iri: "http://www.wikidata.org/entity/Q515")
    country:    _@href(iri: "http://www.wikidata.org/prop/direct/P17")
    SovereignState:    _@href(iri: "http://www.wikidata.org/entity/Q3624078")
}

type ChemicalElement @service(id:"wikidata-sparql") {
    chemicalFormula: String! @service(id:"wikidata-sparql")
    label: [String] @service(id:"wikidata-sparql")
    boilingPoint: String @service(id:"wikidata-sparql")
    meltingPoint: String @service(id:"wikidata-sparql")
    density: String @service(id:"wikidata-sparql")
    discoverer: [Human] @service(id:"wikidata-sparql")
}

type Human @service(id:"wikidata-sparql"){
    placeBirth: [City] @service(id:"wikidata-sparql")
    label: [String] @service(id:"wikidata-sparql")
}

type City @service(id:"wikidata-sparql"){
    country: [SovereignState] @service(id:"wikidata-sparql")
    label: String @service(id:"wikidata-sparql")
}

type SovereignState @service(id:"wikidata-sparql"){
    label: [String] @service(id:"wikidata-sparql")
}
\end{lstlisting}
\end{minipage}

Once an HyperGraphQL instance is set up, we can write the GraphQL query in the provided graphiql interface that can simply be run in a browser. After the query is executed, the instance fetches the results from Wikidata and displays them in the interface. 

HyperGraphQL translates the root node to be an rdf:type of some subject. Since this does not comply with the Wikidata data model, we had a similar issue like the one we discussed in the Fragment based Solution 3 of the previous section regarding inline fragments in GraphQL-LD. To overcome this issue we had to update the source code so that the generated SAPRQL queries use instance of property instead of rdf:type.

The generated SPARQL queries are not viewable by default. We had to modify the logging option in the source code to log SPARQL queries on the command line. Listings XYZ and XYZ show the generated SPARQL query and the results fetched. As before we show only the first 3 results. 

In HyperGraphQL, the fields are converted into optional SPARQL triple patterns. Hence, it might be possible to have an empty array in the results.

\begin{minipage}{\linewidth}
\begin{lstlisting}[label=listing:listing32, caption={Generated SPARQL Query}]
SELECT * WHERE { 
	{ 
		SELECT ?x_1 WHERE { 
			?x_1 <http://www.wikidata.org/prop/direct/P31> <http://www.wikidata.org/entity/Q11344> . 
		}  
	}  
	OPTIONAL { 
		?x_1 <http://www.w3.org/2000/01/rdf-schema#label> ?x_1_1 .
		FILTER (lang(?x_1_1) = "en") .  
	}  
	OPTIONAL { 
		?x_1 <http://www.wikidata.org/prop/direct/P2101> ?x_1_2 . 
	}  
	OPTIONAL { 
		?x_1 <http://www.wikidata.org/prop/direct/P2054> ?x_1_3 . 
	}  
	OPTIONAL { 
		?x_1 <http://www.wikidata.org/prop/direct/P2102> ?x_1_4 . 
	}  
	OPTIONAL { 
		?x_1 <http://www.wikidata.org/prop/direct/P61> ?x_1_5 .
		?x_1_5 <http://www.wikidata.org/prop/direct/P31> <http://www.wikidata.org/entity/Q5> . 
		OPTIONAL { 
			?x_1_5 <http://www.w3.org/2000/01/rdf-schema#label> ?x_1_5_1 .
			FILTER (lang(?x_1_5_1) = "en") .  
		}  
		OPTIONAL { 
			?x_1_5 <http://www.wikidata.org/prop/direct/P19> ?x_1_5_2 .
			?x_1_5_2 <http://www.wikidata.org/prop/direct/P31> <http://www.wikidata.org/entity/Q515> . 
			OPTIONAL { 
				?x_1_5_2 <http://www.w3.org/2000/01/rdf-schema#label> ?x_1_5_2_1 .
				FILTER (lang(?x_1_5_2_1) = "en") .  
			}  
			OPTIONAL { 
				?x_1_5_2 <http://www.wikidata.org/prop/direct/P17> ?x_1_5_2_2 .
				?x_1_5_2_2 <http://www.wikidata.org/prop/direct/P31> <http://www.wikidata.org/entity/Q3624078> . 
				OPTIONAL { 
					?x_1_5_2_2 <http://www.w3.org/2000/01/rdf-schema#label> ?x_1_5_2_2_1 .
					FILTER (lang(?x_1_5_2_2_1) = "en") .  
				}  
			}  
		}  
	}  
	OPTIONAL { 
		?x_1 <http://www.wikidata.org/prop/direct/P274> ?x_1_6 . 
	}  
}
\end{lstlisting}
\end{minipage}

\begin{minipage}{\linewidth}
\begin{lstlisting}[label=listing:listing33, caption={Output}]
{
  "extensions": {},
  "data": {
    "@context": {
      "placeBirth": "http://www.wikidata.org/prop/direct/P19",
      "country": "http://www.wikidata.org/prop/direct/P17",
      "density": "http://www.wikidata.org/prop/direct/P2054",
      "chemicalFormula": "http://www.wikidata.org/prop/direct/P274",
      "_type": "@type",
      "_id": "@id",
      "label": "http://www.w3.org/2000/01/rdf-schema#label",
      "discoverer": "http://www.wikidata.org/prop/direct/P61",
      "meltingPoint": "http://www.wikidata.org/prop/direct/P2101",
      "ChemicalElement_GET": "http://hypergraphql.org/query/ChemicalElement_GET",
      "boilingPoint": "http://www.wikidata.org/prop/direct/P2102"
    },
    "ChemicalElement_GET": [
      {
        "_id": "http://www.wikidata.org/entity/Q876",
        "label": [
          "selenium"
        ],
        "chemicalFormula": "[Se]",
        "boilingPoint": "[1265]",
        "meltingPoint": "[392]",
        "density": "[4.28]",
        "discoverer": [
          {
            "_id": "http://www.wikidata.org/entity/Q353490",
            "label": [
              "Johan Gottlieb Gahn"
            ],
            "placeBirth": []
          },
          {
            "_id": "http://www.wikidata.org/entity/Q151911",
            "label": [
              "J\"{o}ns Jacob Berzelius"
            ],
            "placeBirth": []
          }
        ]
      },
      {
        "_id": "http://www.wikidata.org/entity/Q54377",
        "label": [
          "unbihexium"
        ],
        "chemicalFormula": "[]",
        "boilingPoint": "[]",
        "meltingPoint": "[]",
        "density": "[]",
        "discoverer": []
      },
      {
        "_id": "http://www.wikidata.org/entity/Q743",
        "label": [
          "tungsten"
        ],
        "chemicalFormula": "[W]",
        "boilingPoint": "[5930, 10701]",
        "meltingPoint": "[6170, 3410]",
        "density": "[19.3]",
        "discoverer": [
          {
            "_id": "http://www.wikidata.org/entity/Q182745",
            "label": [
              "Fausto Elhuyar"
            ],
            "placeBirth": []
          },
          {
            "_id": "http://www.wikidata.org/entity/Q728386",
            "label": [
              "Juan Jos\'{e} Elhuyar"
            ],
            "placeBirth": []
          }
        ]
      },
	  ...
      }
    ]
  },
  "errors": []
}
\end{lstlisting}
\end{minipage}


\section{Technicalities and setup}
There were some challenges encountered when trying to test GraphQL-LD and HypergraphQL, and using them to query Wikidata. 

With GraphQL-LD, there were issues running the examples provided in the official documentation. These were related with compatibility when invoking the programmatic API written in ES6 on modules written in CommonJS. To solve this issue we had to update the API codes and the root level node package. Since GraphQL-LD is predicate based, it was challenging to find a workaround to query Wikidata. There were no examples on the documentation for subject-based querying. A significant amount of time and research was invested in finding out the three proposed solutions. The process of creating a default context required a lot of time and resource consumption. The original dump file of truthy statements was around 60GB, and hence required a large bandwidth and memory consumption for downloading and parsing the file. 

An important part of our work is to analyze the generated SPARQL queries. When working with HyperGraphQL, the viewing of the generated SPARQL queries is not available by default. We had to study the source code in detail in order to log them on the command line. 

We discussed in the previous chapter about the data model in Wikidata where the RDF property name "rdf:type" is replaced by Wikidata property "instance of". This property is used when the subject is an instance of a class. In GraphQL-LD, the fragments in GraphQL queries apply on types. When translating to SPARQL queries, the predicate rdf:type is used to connect the subject to some type of object. We had to analyze the "GraphQL to SPARQL algebra" module in order to update the source code so that GraphQL-LD used "instance of" instead of "rdf:type". Similarly, HyperGraphQL also used rdf:type as the property when translating the root node in GraphQL queries to SPARQL. A similar effort was needed to update the source code such that "instance of" was used instead of "rdf:type" so that data from Wikidata could be queried.

We have created a public repository\footnote{https://github.com/AnasShahab/querying-wikidata-with-graphQL.} where all the changes made to the source code of both the approaches, GraphQL-LD and HyperGraphQL, are retained. This repository can be used to test the querying of Wikidata via both of the approaches. Along with containing the installation guide to use the repository, it contains examples to query Wikidata including the ones used in this report.

\pagebreak

\chapter{Comparison of GraphQL-LD and HyperGraphQL}
\label{ch:5}
In this chapter, we compare GraphQL-LD and HyperGraphQL by highlighting the important similarities and differences they have among each other. At the end of the chapter, we provide tables that summarize the comparison against the features of GraphQL and SPARQL respectively.

The comparison is based on the following criteria:
\begin{itemize}
\item Implementation specific details such as usage of schema.
\item Standard GraphQL features such as passing arguments into fields.
\item Important SPARQL features such aggregates and values.
\end{itemize}

We provide examples (when applicable) to demonstrate the comparison, where each example consists of the following:

\begin{itemize}
\item A SPARQL query taken from the official Wikidata documentation\footnote{https://www.wikidata.org/wiki/Wikidata:SPARQL\_query\_service/queries/examples}.
\item A GraphQL query in GraphQL-LD accompanied with the corresponding generated SPARQL query.
\item A GraphQL query in HyperGraphQL accompanied with the corresponding generated SPARQL query.
\end{itemize}

For our examples we use the the JSON-LD context (used by GraphQL-LD) and the annotated schema (used by HyperGraphQL) shown in Listings~\ref{lst:35} and \ref{lst:36} respectively. For writing GraphQL queries in GraphQL-LD we use the \textit{ID based} solution.

\begin{minipage}{\linewidth}
\begin{lstlisting}[label=lst:35, caption={JSON-LD Context in GrahQL-LD}, language=GraphQL]
{
  "@context": {
  	"@vocab":"http://www.wikidata.org/entity/", 
  	"wd": "http://www.wikidata.org/entity/",
  	"wdt" :"http://www.wikidata.org/prop/direct/",
  	"instance": "wdt:P31",
  	"cat": "wd:Q146",
  	"label": "http://www.w3.org/2000/01/rdf-schema#label",
  	"chemicalCompound": "wd:Q11173",
  	"color": "wdt:P462",
  	"rgb": "wdt:P465",
  	"JohannSebastianBach": "wd:Q1339",
  	"hasFather": "wdt:P22",
  	"fatherOf": { "@reverse":"wdt:P22" },
  	"hasMother": "wdt:P25",
  	"cathedral": "wd:Q2977",
  	"place": "wdt:P131",
  	"Paris": "wd:Q90"
  }
}
\end{lstlisting}
\end{minipage}

\begin{minipage}{\linewidth}
\begin{lstlisting}[label=lst:36, caption={Annotated Scehma in HyperGraphQL}, language=GraphQL]
type __Context {
  Cat:   _@href(iri: "http://www.wikidata.org/entity/Q146")
  label:  _@href(iri: "http://www.w3.org/2000/01/rdf-schema#label")
  ChemicalCompound: _@href(iri: "http://www.wikidata.org/entity/Q11173")
  color: _@href(iri: "http://www.wikidata.org/prop/direct/P462")
  Color: _@href(iri: "http://www.wikidata.org/entity/Q1075")
  rgb: _@href(iri: "http://www.wikidata.org/prop/direct/P465")
  Airport:_@href(iri: "http://www.wikidata.org/entity/Q1248784")
  connectingLine: _@href(iri: "http://www.wikidata.org/prop/direct/P81")
  placeServed: _@href(iri: "http://www.wikidata.org/prop/direct/P931")
  City: _@href(iri: "http://www.wikidata.org/entity/Q515")
  coordinateLocation: _@href(iri: "http://www.wikidata.org/prop/direct/P625")
}

type Cat @service(id:"wikidata-sparql") {
  label: [String] @service(id:"wikidata-sparql")
}

type ChemicalCompound @service(id:"wikidata-sparql") {
  label: [String] @service(id:"wikidata-sparql")
  color: [Color] @service(id:"wikidata-sparql")
}

type Color @service(id:"wikidata-sparql") {
  label: [String] @service(id:"wikidata-sparql")
  rgb: [String] @service(id:"wikidata-sparql")
}

type Airport @service(id:"wikidata-sparql") {
  label: [String] @service(id:"wikidata-sparql")
  connectingLine: [Airport] @service(id:"wikidata-sparql")
  placeServed: [City] @service(id:"wikidata-sparql")
}

type City @service(id:"wikidata-sparql") {
  label: [String] @service(id:"wikidata-sparql")
  coordinateLocation: [String] @service(id:"wikidata-sparql")
}
\end{lstlisting}
\end{minipage}


\section{Schema}
HyperGraphQL uses schemas for querying. This is similar to the original GraphQL specificiations. The schema maps every type and relationship to an unique IRI that points to a resource in the RDF source. The advantage of using a schema is that we can limit data that the user is allowed to query. 

However, this also means that some restriction in inevitable since we have to mention the type of every object node that is a resource. For example, in Wikidata the cat labelled \textit{Orangey} \texttt{(Q677525}) is linked to several items through the property \textit{present in work} \texttt{(P1441)}. Each of these items are an instance of either a \textit{film} or a \textit{television series} but not both. When linking the property present in work to an item in the schema, we have to either define that item to be a \textit{film} or \textit{television series}, hence the limitation. Also, there are cases where we want all the nodes that have a particular relationship between them, and do not specify explicitly the type that each of the node belongs to. This is not possible in HyperGraphQL since we have to define for every resource a type. Moreover, every type and field in the schema can only be associated with one service (Chapter 3 Section~\ref{subsec:configuration}). 

On the other hand, GraphQL-LD operates schema-less. It translates a GraphQL query into a SPARQL query aided by a JSON-LD context. This context is responsible for mapping the GraphQL definitions to unique IRIs. The benefit of using no schema is that GraphQL is more flexible in the sense that RDF sources can be queried in the same fashion as one would use SPARQL queries against the endpoint. This removes the restrictions on the data that can be queried, and allows to simple query and relationships between nodes without mentioning the type of the nodes.

Since HyperGraphQL uses a schema, another advantage is that it distinguishes what should be queried as a type and what should be treated as a property~\cite{Werbrouck2019a}. On the other hand, since GraphQL is predicate based, there is no direct way to distinguish this. We have to use workarounds to query for specific class.


\section{Intermediary server}
We create and run a HyperGraphQL instance by setting up an intermediary server using a configuration file and an annotated schema. A GraphQL endpoint is exposed that points to the service or Linked Data source configured in the parameters of the configuration file. However, when one needs to change the service, it required to stop the running intermediary server and create a new configuration file. Additionally, the annotated schema also required to be changed to comply with the new service. 

On the other hand, GraphQL-LD does not require the setup of a server. It can be executed simply via the command line by providing a GraphQL query, JSON-LD context and SPARQL endpoint. This implementation is relatively easier and provides more flexibility when one needs to query a different Linked Data source. 


\section{Updating data}
Updating existing data is an important feature when working with knowledge graphs. GraphQL implements this using the mutation operation type. Mutations allow the creation, update and deletion of data. SPARQL implements creation and deletion using the operations \texttt{INSERT} and \texttt{DELETE} respectively, and update using a combination of the two operations. However, neither GraphQL-LD nor HyperGraphQL currently support mutations.  


\section{Filter Labels and Descriptions}
When querying Wikidata we can use the specialized service with \texttt{wikibase:label}\footnote{short for \texttt{<http://wikiba.se/ontology\#label>}} in SPARQL queries to fetch labels and descriptions in specific languages. Alternatively, we can also use the \texttt{rdf:label} and \texttt{schema:description}\footnote{short for \texttt{<http://schema.org/description>}} properties along with a \texttt{FILTER} function as is common when querying RDF graphs. 

We cannot use the specialized service in any of the two approaches. Instead the \texttt{rdf:label} and \texttt{schema:description} properties are used. GraphQL-LD does not support the use of filter, and hence we cannot fetch labels in a particular language. On the other hand, in HyperGraphQL we can specify the language of the fetched string (Chapter 3 Secion\ref{subsec:annotated-schema}). The examples shown in this chapter demonstrate this.

\section{Reverse Querying}
\label{sec:reverse}
GraphQL-LD supports reversing the direction of a property by the use of \texttt{@reverse} context option in the JSON-LD context. Standard GraphQL specifications do not have this feature. Reverse fields are useful in cases where we want to reverse the relationship between a parent and a child node. This is similar to the inverse path feature of property paths that reverses the direction of edge traversal.\footnote{https://www.w3.org/TR/sparql11-query/\#propertypaths} HyperGraphQL does not have this feature.

Listing~\ref{lst:37} shows a SPARQL query that fetches the siblings of Johann Sebastian Bach using the inverse path feature.

\begin{minipage}{\linewidth}
\begin{lstlisting}[label=lst:37, caption={SPARQL query showing usage of inverse path}, language=SPARQL]
SELECT ?sibling ?siblingLabel
WHERE
{
  wd:Q1339 wdt:P22 ?father.
  ?father ^wdt:P22 ?sibling.
  SERVICE wikibase:label { bd:serviceParam wikibase:language "en". }
}
\end{lstlisting}
\end{minipage}

We can generate a similar query in GraphQL-LD. The JSON-LD context that we had defined in Listing XYZ has \textit{hasFather} and \textit{ofFather} entries, both of which point to the IRI pointing to father property in Wikidata. However, the \textit{ofFather} entry has the \texttt{@reverse} keyword added in the JSON-LD context that reverse the direction of the property. We use \texttt{(\_:JohannSebastianBach)} as an argument in id to define the id to be \textit{JohannSebastianBach}. This is how we can set a \textit{block scoped id} in GraphQL-LD.\footnote{https://github.com/rubensworks/graphql-to-sparql.js\#setting-a-block-scoped-id}

Listing~\ref{lst:38} shows the GraphQL query and generated SPARQL query in GraphQL-LD.

\begin{minipage}{\linewidth}
\begin{lstlisting}[columns=fullflexible, label=lst:38, caption={GraphQL query and generated SPARQL query in GraphQL-LD}, language=SPARQL]
query {
  id(_: JohannSebastianBach)
  hasFather {
  	fatherOf {
  	  id
      label
    }
  }
}


SELECT ?hasFather_fatherOf_id ?hasFather_fatherOf_label WHERE {
  <http://www.wikidata.org/entity/Q1339> <http://www.wikidata.org/prop/direct/P22> ?hasFather.
  ?hasFather_fatherOf_id <http://www.wikidata.org/prop/direct/P22> ?hasFather;
    <http://www.w3.org/2000/01/rdf-schema#label> ?hasFather_fatherOf_label.
}
\end{lstlisting}
\end{minipage}


\section{Optional Fields}

In GraphQL-LD, GraphQL query fields are non-null by default. They can be defined nullable using the \texttt{@optional} directive. This belongs to one of the custom directives provided by GraphQL-LD and is not implemented in the standard specifications of GraphQL. In SPARQL we use the corresponding \texttt{OPTIONAL} feature in triple patterns.

HyperGraphQL always translates GraphQL fields into \texttt{OPTIONAL} SPARQL triple patterns. GraphQL allows to define fields to be non-null in the schema. However, HyperGraphQL does not support this feature and we cannot implement it in its annotated schema. As a result of this limitation, we get more results than expected in HyperGraphQL when writing GraphQL queries. 

Listing~\ref{lst:39} shows a SPARQL query that fetches the colours of chemical compounds and if possible the RGB of the colours.

\begin{minipage}{\linewidth}
\begin{lstlisting}[label=lst:39, caption={SPARQL query showing usage of OPTIONAL feature}, language=SPARQL]
SELECT ?compound ?color ?rgb  WHERE
{
  ?compound wdt:P31 wd:Q11173;
            wdt:P462 ?color.
  OPTIONAL { ?color wdt:P465 ?rgb. }
  SERVICE wikibase:label { bd:serviceParam wikibase:language "en". }
}
\end{lstlisting}
\end{minipage}

Listings~\ref{lst:40} and \ref{lst:41} show the GraphQL query and generated SPARQL query in GraphQL-LD and HyperGraphQL respectively.

\begin{minipage}{\linewidth}
\begin{lstlisting}[columns=fullflexible, label=lst:40, caption={GraphQL query and generated SPARQL query in GraphQL-LD}, language=SPARQL]
query {
  id(instance:compound) {
    id
    label
    color {
      id
      label
      rgb @optional
    }
  }
}


SELECT ?id ?id_id ?id_label ?id_color_id ?id_color_label ?id_color_rgb WHERE {
  ?id_id <http://www.wikidata.org/prop/direct/P31> <http://www.wikidata.org/entity/Q11173>;
  %\phantom{?id\_id }%<http://www.w3.org/2000/01/rdf-schema#label> ?id_label;
  %\phantom{?id\_id }%<http://www.wikidata.org/prop/direct/P462> ?id_color_id.
  ?id_color_id <http://www.w3.org/2000/01/rdf-schema#label> ?id_color_label.
  OPTIONAL { ?id_color_id <http://www.wikidata.org/prop/direct/P465> ?id_color_rgb. }      
}
\end{lstlisting}
\end{minipage}



\begin{minipage}{\linewidth}
\begin{lstlisting}[columns=fullflexible, label=lst:41, caption={GraphQL query and generated SPARQL query in HyperGraphQL}, language=SPARQL]
{
  ChemicalCompound_GET {
    _id
    label(lang: "en")
    color {
      _id
      label(lang: "en")
      rgb
    }
  }
}


SELECT * WHERE { 
  { 
    SELECT ?x_1 WHERE { 
      ?x_1 <http://www.wikidata.org/prop/direct/P31> <http://www.wikidata.org/entity/Q11173> . 
    }  
  }  
  OPTIONAL { 
    ?x_1 <http://www.w3.org/2000/01/rdf-schema#label> ?x_1_1 .
    FILTER (lang(?x_1_1) = "en") .  
  }  
  OPTIONAL { 
    ?x_1 <http://www.wikidata.org/prop/direct/P462> ?x_1_2 .
    ?x_1_2 <http://www.wikidata.org/prop/direct/P31> <http://www.wikidata.org/entity/Q1075> . 
    OPTIONAL { 
      ?x_1_2 <http://www.w3.org/2000/01/rdf-schema#label> ?x_1_2_1 .
      FILTER (lang(?x_1_2_1) = "en") .  
    }  
    OPTIONAL { 
      ?x_1_2 <http://www.wikidata.org/prop/direct/P465> ?x_1_2_2 . 
    }  
  }  
}
\end{lstlisting}
\end{minipage}


\section{LIMIT and OFFSET}

In GraphQL-LD we can define the maximum number of results for a field with the \texttt{first} argument. Additionally, we can specify the index of the first result using the \texttt{offset} argument. Limits and offsets are part of pagination technique in GraphQL. These are similar to \texttt{LIMIT} and \texttt{OFFSET} in SPARQL.

In HyperGraphQL we can pass \texttt{limit} and \texttt{offset} arguments to every \textit{TypeName\_GET} query field.

Listing~\ref{lst:42} shows a SPARQL query to fetch items whose value of \texttt{instance of} is a house cat. The \texttt{LIMIT} is set to \texttt{10} and \texttt{OFFSET} to \texttt{2}.


\begin{minipage}{\linewidth}
\begin{lstlisting}[label=lst:42, caption={SPARQL query showing usage of LIMIT and OFFSET}, language=SPARQL]
SELECT ?item ?itemLabel
WHERE
{
    ?item wdt:P31 wd:Q146;.
    SERVICE wikibase:label { bd:serviceParam wikibase:language "[AUTO_LANGUAGE],en". }
} OFFSET 2 LIMIT 10
\end{lstlisting}
\end{minipage}

Listings~\ref{lst:43} and \ref{lst:44} show the GraphQL query and generated SPARQL query in GraphQL-LD and HyperGraphQL respectively.


\begin{minipage}{\linewidth}
\begin{lstlisting}[columns=fullflexible, label=lst:43, caption={GraphQL query and generated SPARQL query in GraphQL-LD}, language=SPARQL]
query {
  id(instance:cat first: 10 offset: 2) {
    id
    label
  }
}


SELECT ?id ?id_id ?id_label WHERE {
  SELECT ?id_id ?id_label WHERE {
    ?id_id <http://www.wikidata.org/prop/direct/P31> <http://www.wikidata.org/entity/Q146>;
    %\phantom{?id\_id }%<http://www.w3.org/2000/01/rdf-schema#label> ?id_label.
  }
  OFFSET 2
  LIMIT 10
}
\end{lstlisting}
\end{minipage}


\begin{minipage}{\linewidth}
\begin{lstlisting}[columns=fullflexible, label=lst:44, caption={GraphQL query and generated SPARQL query in HyperGraphQL}, language=SPARQL]
{
  Cat_GET(limit: 10, offset: 2) {
    _id
    label(lang: "en")
  }
}


SELECT * WHERE {
  {
    SELECT ?x_1 WHERE {
      ?x_1 <http://www.wikidata.org/prop/direct/P31> <http://www.wikidata.org/entity/Q146> .
    } LIMIT 10 OFFSET 2
  }
  OPTIONAL {
    ?x_1 <http://www.w3.org/2000/01/rdf-schema#label> ?x_1_1 .
    FILTER (lang(?x_1_1) = "en") .
  }
}
\end{lstlisting}
\end{minipage}


\section{COUNT Aggregate Function}

In GraphQL-LD we can use the \texttt{totalCount} field to find the total number of items that match a certain query. This is another pagination technique used in GraphQL and is useful to get the total numbers of results available irrespective of any limit or offset fields. This is similar to the \texttt{COUNT} aggregate function in SPARQL that counts the number of triples or other query elements that match a particular pattern. 

HyperGraphQL does not support the totalCount feature.

Listing~\ref{lst:45} shows a SPARQL query that returns the count of the total number of humans listed in Wikidata.


\begin{minipage}{\linewidth}
\begin{lstlisting}[label=lst:45, caption={SPARQL query showing usage of COUNT}, language=SPARQL]
SELECT (COUNT(*) AS ?count)
WHERE {
  ?item wdt:P31 wd:Q5 .
}
\end{lstlisting}
\end{minipage}

Listing\ref{lst:46} shows the GraphQL query and generated SPARQL query in GraphQL-LD. 

\begin{minipage}{\linewidth}
\begin{lstlisting}[label=lst:46, caption={GraphQL query and generated SPARQL query in GraphQL-LD}, language=SPARQL]
query {
  id(instance: human){
    totalCount
  }
}


SELECT ?id ?id_totalCount WHERE { 
  SELECT (COUNT(?id) AS ?id_totalCount) WHERE { 
    ?id <http://www.wikidata.org/prop/direct/P31> <http://www.wikidata.org/entity/Q5>. 
  } 
}
\end{lstlisting}
\end{minipage}


\section{VALUES}

SPARQL uses the \texttt{VALUES} keyword to specify a set of values for one or more variables that are used as inputs to queries. GraphQL-LD does not support this feature. 

HyperGraphQL exposes the query field \textit{TYPENAME\_GET\_BY\_ID} for every type in the annotated schema. This field takes a list of instances of that particular type and uses the \texttt{VALUES} keyword in the generated SPARQL query to inject the list as input into the subject node of the triple pattern - \texttt{?subject rdf:type \textit{<IRI of type>}}, where IRI of type is the IRI that points to the \textit{TYPENAME}. At the moment only injection of values in subject nodes of such triple patterns is supported.

Listing~\ref{lst:47} shows a SPARQL query that fetches the destinations that can be reached from Antwerp International Airport. Antwerp International Airport is passed into the \textit{?airport} variables using the \texttt{VALUE} keyword.


\begin{minipage}{\linewidth}
\begin{lstlisting}[label=lst:47, caption={SPARQL query showing usage of VALUES}, language=SPARQL]
SELECT ?connectsairport ?connectsairportLabel ?place_served ?place_servedLabel ?coor
WHERE
{
  VALUES ?airport { wd:Q17480 }  
  ?airport wdt:P81 ?connectsairport ;
           wdt:P625 ?base_airport_coor .
  ?connectsairport wdt:P931 ?place_served ;
                   wdt:P625 ?coor .
  SERVICE wikibase:label { bd:serviceParam wikibase:language "en". }
}
\end{lstlisting}
\end{minipage}

Listing~\ref{lst:48} shows the GraphQL query and generated SPARQL query in HyperGraphQL. 

\begin{minipage}{\linewidth}
\begin{lstlisting}[columns=fullflexible, label=lst:48, caption={GraphQL query and generated SPARQL query in HyperGraphQL}, language=SPARQL]
{
  Airport_GET_BY_ID(uris: ["http://www.wikidata.org/entity/Q17480"]) {
    _id
    label(lang: "en")
    connectingLine {
      _id
      label(lang: "en")
      placeServed {
        _id
        	label(lang: "en")
        	coordinateLocation
      }
    }
  }
} 


SELECT * WHERE {
  VALUES ?x_1 { <http://www.wikidata.org/entity/Q17480> }
  ?x_1 <http://www.wikidata.org/prop/direct/P31> <http://www.wikidata.org/entity/Q1248784> .
  OPTIONAL {
    ?x_1 <http://www.wikidata.org/prop/direct/P81> ?x_1_1 .
    ?x_1_1 <http://www.wikidata.org/prop/direct/P31> <http://www.wikidata.org/entity/Q1248784> .
    OPTIONAL {
      ?x_1_1 <http://www.w3.org/2000/01/rdf-schema#label> ?x_1_1_1 .
      FILTER (lang(?x_1_1_1) = "en") .
    }
    OPTIONAL {
      ?x_1_1 <http://www.wikidata.org/prop/direct/P931> ?x_1_1_2 .
      ?x_1_1_2 <http://www.wikidata.org/prop/direct/P31> <http://www.wikidata.org/entity/Q515> .
      OPTIONAL {
        ?x_1_1_2 <http://www.w3.org/2000/01/rdf-schema#label> ?x_1_1_2_1 .
        FILTER (lang(?x_1_1_2_1) = "en") .
      }
      OPTIONAL {
        ?x_1_1_2 <http://www.wikidata.org/prop/direct/P625> ?x_1_1_2_2 .
      }
    }
  }
  OPTIONAL {
    ?x_1 <http://www.w3.org/2000/01/rdf-schema#label> ?x_1_2 .
    FILTER (lang(?x_1_2) = "en") .
  }
}
\end{lstlisting}
\end{minipage}

\section{Arguments}

In GraphQL we can pass arguments to pass data to every field or object, where object is a type in the GraphQL schema. This is in addition to the predefined arguments such as \texttt{first} and \texttt{offset}. GraphQL-LD also allows passing arguments into fields or objects. These are converted into subject or object nodes of triples in SPARQL.

HyperGraphQL does not allow users to pass arguments into fields or objects other than the predefined ones. (See Chapter 3 Section~\ref{subsec:annotated-schema})

Listing~\ref{lst:49} shows a SPARQL query used to fetch the cathedrals in Paris. 


\begin{minipage}{\linewidth}
\begin{lstlisting}[label=lst:49, caption={SPARQL query to fetch cathedrals in Paris}, language=SPARQL]
SELECT ?item ?itemLabel ?placeLabel
WHERE
{
  ?item wdt:P31 wd:Q2977 .
  ?item wdt:P131 ?place .
  ?place wdt:P131 wd:Q90 .
  SERVICE wikibase:label { bd:serviceParam wikibase:language "en" . }
}
\end{lstlisting}
\end{minipage}

Listing~\ref{lst:50} shows the GraphQL query and generated SPARQL query in GraphQL-LD. The argument (place: Paris) is passed as a parameter to the id field of the place object.

\begin{minipage}{\linewidth}
\begin{lstlisting}[columns=fullflexible, label=lst:50, caption={GraphQL query and generated SPARQL query in GraphQL-LD}, language=SPARQL]
query {
  id(instance: cathedral){
    id
    label
    place{
      id (place: Paris)
      label
    }
  }
}


SELECT ?id ?id_id ?id_label ?id_place_id ?id_place_label WHERE {
  ?id_id <http://www.wikidata.org/prop/direct/P31> <http://www.wikidata.org/entity/Q2977>;
         <http://www.w3.org/2000/01/rdf-schema#label> ?id_label;
         <http://www.wikidata.org/prop/direct/P131> ?id_place_id.
  ?id_place_id <http://www.wikidata.org/prop/direct/P131> <http://www.wikidata.org/entity/Q90>;
               <http://www.w3.org/2000/01/rdf-schema#label> ?id_place_label.
}
\end{lstlisting}
\end{minipage}


\section{Alternative Fields}

In GraphQL-LD we can use multiple fields for retrieving a value via the \texttt{alt} argument in fields. Multiple alternatives can be defines by inserting them in a list: \texttt{\textit{field}(alt: [alt1, alt2, alt3, …])}. This is similar to matching alternative paths in property paths using the vertical bar symbol to always match a path of length one. We use this when we want the path to use any one of the properties in the alternative paths. 

HyperGraphQL does not support defining alternative fields.

Listing~\ref{lst:51} shows a query to fetch father or mother of Johann Sebastian Bach.


\begin{minipage}{\linewidth}
\begin{lstlisting}[columns=fullflexible, label=lst:51, caption={SPARQL query showing the usage of alternative paths}, language=SPARQL]
SELECT ?parent ?parentLabel
WHERE
{
  wd:Q1339 (wdt:P22|wdt:P25) ?parent.
  SERVICE wikibase:label { bd:serviceParam wikibase:language "en". }
}
\end{lstlisting}
\end{minipage}

Listing~\ref{lst:52} shows the GraphQL query and generated SPARQL query in GraphQL-LD. We used the parent alias to represent a father or mother.

\begin{minipage}{\linewidth}
\begin{lstlisting}[label=lst:52, caption={GraphQL query and generated SPARQL query in GraphQL-LD}, language=SPARQL]
query {
  id(_: JohannSebastian)
  parent: hasFather(alt: hasMother){
    id
    label
  }
}


SELECT ?parent_id ?parent_label WHERE {
  ?parent_id <http://www.w3.org/2000/01/rdf-schema#label> ?parent_label.
  <http://www.wikidata.org/entity/Q1339> (<http://www.wikidata.org/prop/direct/P22>|
  <http://www.wikidata.org/prop/direct/P25>) ?parent_id.
}
\end{lstlisting}
\end{minipage}


\section{Default vocabulary}

We can set a default vocabulary (namespace) for all values of keys in a JSON-LD context file with the \texttt{@vocab} keyword. In the JSON-LD context file in Listing~\ref{lst:35} we set the default vocabulary to \texttt{http://www.wikidata.org/entity/} by \texttt{@vocab": "http://www.wikidata.org/entity/}.

GraphQL-LD can make use of the default vocabulary as it uses a JSON-LD context. Since HypergraphQL does not use JSON-LD context, this cannot be used in its implementation. 

Listing~\ref{lst:53} shows a GraphQL query in GraphQL-LD and the corresponding generated SPARQL query that fetches the names of items that are instances of humans in Wikidata. We do not have a key for the argument human. As a result, human will have the IRI \texttt{http://www.wikidata.org/entity/human}. 


\begin{minipage}{\linewidth}
\begin{lstlisting}[label=lst:53, caption={GraphQL query and generated SPARQL query in GraphQL-LD}, language=SPARQL]
query {
  id(instance: human){
    name
  }
}

SELECT ?id ?id_name WHERE {
  ?id <http://www.wikidata.org/prop/direct/P31> <http://www.wikidata.org/entity/human>;
      <http://www.wikidata.org/prop/direct/P735> ?id_name.
}
\end{lstlisting}
\end{minipage}


From the example we see that human is appended at the end of the default vocabulary. The same IRI is generated for human when it has an entry in the JSON-LD context but does not have a value. Moreover, if human has a non-IRI value then that value is appended at the end of the default vocabulary. 

When querying Wikidata, we use \texttt{http://www.wikidata.org/entity/} to represent entity nodes and \texttt{http://www.wikidata.org/prop/direct/} for predicates of truthy statements. We can only set one namespace as the default vocabulary. Since items are significantly larger in number than properties, setting the vocabulary to \texttt{http://www.wikidata.org/entity/} is a good option as it might not be possible to define all items in a JSON-LD context.



Table~\ref{tab:2} provides a comparison of the two approaches with the GraphQL features (See Chapter 2, Section~\ref{sec:graphql}).


\begin{table}[h]
	\begin{center}
		\caption{Comparison of the two approaches with GraphQL features}
		\label{tab:2}
		\renewcommand{\arraystretch}{2}
		\begin{tabular}{ccc}
		

			\toprule
			
			& \textbf{GraphQL-LD} & \textbf{HyperGraphQL}  \\ 
		
			\midrule
			
			Arguments & \cmark & \cmark \tablefootnote{only defined ones}	 \\
			
			Aliases & \cmark & \cmark  \\ 
			
			Fragments & \cmark & \cmark \tablefootnote{\cite{Werbrouck2019a} mentions possible but we could not implement it}  \\ 
			
			Inline Fragments & \cmark & \cmark \tablefootnote{Same issue as with Fragments}  \\ 
			
			
			
			Variables & \cmark & \cmark  \\ 
			
			Directives & \cmark & \cmark \\ 
			
			Mutations & \xmark & \xmark \\ 
			
			Pagination & \cmark & \cmark \\ 
			
			Introspection & \xmark & \cmark \\ 
			
			Nullability & \cmark & \cmark \tablefootnote{by default} \\ 
				
			\bottomrule

		\end{tabular}
	\end{center}
\end{table}


Table~\ref{tab:3} provides a comparison of the two approaches with some standard SPARQL features.


\begin{table}[h]
	\begin{center}
		\caption{Comparison of the two approaches with some standard SPARQL features}
		\label{tab:3}
		\renewcommand{\arraystretch}{2}
		\begin{tabular}{ccc}
		

			\toprule
			
			& \textbf{GraphQL-LD} & \textbf{HyperGraphQL}  \\ 
		
			\midrule
			
			LIMIT & \cmark & \cmark	 \\
			
			OFFSET & \cmark & \cmark  \\ 
			
			ORDER BY & \xmark & \xmark  \\ 
			
			GROUP BY & \xmark & \xmark  \\ 
			
			FILTER & \xmark & \cmark \tablefootnote{only for lang labels}  \\ 
			
			ASK/CONSTRUCT/DESCRIBE & \xmark & \xmark \\ 
			
			DISTINCT & \xmark & \xmark \\ 
			
			UNION & \xmark & \xmark \\ 
			
			MINUS & \xmark & \xmark \\ 
			
			OPTIONAL & \cmark & \cmark \\ 
			
			VALUES & \xmark & \cmark \tablefootnote{only for subject queries} \\ 
			
			BIND & \xmark & \xmark \\ 	
			
			COUNT & \cmark & \xmark \\ 			
			
			Other Aggregates & \xmark & \xmark \\ 	
			
			HAVING & \xmark & \xmark \\ 	
			
			Reverse Paths & \cmark & \xmark \\ 
			
			Alternative paths & \cmark & \xmark \\ 
			
			
			\bottomrule

		\end{tabular}
	\end{center}
\end{table}


Both GraphQL-LD and HyperGraphQL are potential solutions to querying RDF graphs using GraphQL. They have their own benefits and limitations. Since GraphQL queries tree structures, it is less expressive than SPARQL~\cite{Werbrouck2019}. On the other hand, GraphQL facilitates querying basic RDF data. The two approaches support many standard features that are typical in GraphQL. However, both of the approaches are work in progress and have their own limitations. Hence, not all the expressivity of GraphQL can currently be implemented~\cite{Werbrouck2019}.

When comparing between each other, GraphQL-LD is a more lightweight implementation of GraphQL for querying linked data. It is more suitable when we want to query different datasets by changing our endpoints as it operates without a schema and does not require the setup of a server. It also takes less time to implement it as opposed to HyperGraphQL.

On the other hand, HyperGraphQL is a more powerful implementation. It is preferable when we want to query large, complex datasets such as knowledge graphs like Wikidata. An uniform setup with predefined set of types and fields in the annotated schema makes it stable and friendlier to use~\cite{Werbrouck2019a}. 



\pagebreak

\chapter{Conclusion and Future work}
\label{ch:6}
In this report, we introduced knowledge graphs and their practical uses. We highlighted Wikidata, which is a popular knowledge graph, and has many commercial and research oriented applications. We also discussed about RDF and showed how it compares against the Wikidata data model. The report provided a brief understanding of SPARQL and GraphQL, and highlighted some key differences that exist between them. These included the fundamental goals, commercial usage, expressivity and ability to query multiple data sources.

As part of the work, we researched on several different approaches to query RDF data via GraphQL. These included approaches such as Stardog and TopBraid EDG being commercial solutions, and GraphQL-LD and HyperGraphQL being open-sourced approaches. Since GraphQL-LD and HyperGraphQL are open sourced, they provide the opportunity to query arbitrary RDF data source and make modifications to code. We focused on the fundamental principles and features of the two approaches.

Our focus on this report was on querying Wikidata via GraphQL, and hence we used and adapted GraphQL-LD and HyperGraphQL to work with Wikidata accordingly. As they are open-sourced, we were able to make changes that were necessary for the implementation. GraphQL-LD is predicate-based and a workaround is needed to query for resources that are a type of come specific class. We offered three solutions and offered examples for each. Moreover, we produced a default JSON-LD context for GraphQL-LD that can be used when we no context is provided as input. Both GraphQL-LD and HyperGraphQL were used to query data from Wikidata which was demonstrated using examples.

Lastly, we provided a comparison between GraphQL-LD and HyperGraphQL. We highlighted on schema usage, intermediary server, updating data, reverse querying and generated SPARQL queries. When comparing the generated SPARQL queries we used examples that distinguished the capabilities of the two approaches.
GraphQL is more flexible and developer friendly than SPARQL. It is popular in the web development community. However, SPARQL is more expressive and is specialized to query and manipulate RDF data in the Semantic Web. Using GraphQL to query RDF data makes knowledge graphs like Wikidata more convenient to use.  GraphQL-LD and HyperGraphQL provide a layer of abstraction that hides the complexity of SPARQL and allows us to query RDF data using GraphQL.  

Both GraphQL-LD and HyperGraphQL are work in progress. They do not have all the features that are typical to GraphQL. However, they provide a prospective way to query Linked Data via GraphQL. Updating data is crucial when working with knowledge graphs. Mutations are not supported by any of the two approaches and they allow users to only read data from existing RDF datasets. 

Wikidata has different way of representing the relationships between resources as compared to RDF. Analogously, other knowledge graphs can also have such differences based on the vocabulary used. Currently in both GraphQL-LD and HyperGraphQL, we have to modify the source code whenever we want to switch between different knowledge graphs. This is a limitation when we want to query different knowledge graphs and perform federated querying.

Furthermore, the workarounds we proposed for GraphQL-LD to perform subject-based querying can be further developed to make them more user friendly and intuitive. HyperGraphQL does not allow non-null feature in the GraphQL schema. This is a limitation when querying for optional fields. Moreover, passing arbitrary arguments to fields is not supported in HyperGraphQL.
\lukas{Passing arbitrary arguments is generally not supported by GraphQL, they need to be defined in the schema. GraphQL-LD is kind of special here since it has it's own semantics that handles many arguments.}

In the future, we aim to work on updating GraphQL-LD and HyperGraphQL so that they can more efficiently and productively be used to query knowledge graphs. We wish to provide solutions to the existing limitations and incorporate additional features that are available in standard GraphQL specifications. We hope that in doing so we can use the full potential of knowledge graphs like Wikidata. 

%\appendix
\singlespacing
\bibliographystyle{ieeetr}
\bibliography{main}


%\printglossaries
\end{document}
