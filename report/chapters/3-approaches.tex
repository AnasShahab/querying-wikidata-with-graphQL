\chapter{Approaches for Querying RDF Graphs with GraphQL}

In this chapter, we focus on ways of queries RDF Graphs using GraphQL. In section XYZ of the previous chapter, we highlighted the differences between GraphQL and SPARQL. This gives rise to the idea of using GraphQL to query RDF graphs, thereby overcoming the limitations of using SPARQL directly. 
In recent years, there have been attempts at providing approaches to querying linked data represented by RDF via GraphQL. This gave rise to several commercial and open-source solutions. Most notable ones include:

\begin{itemize}
	\item Stardog
	\item TopBraid EDG
	\item Ontotext Platform
	\item GraphQL-LD
	\item HyperGraphQL
	\item UltraGraphQL

\end{itemize}

Stardog\footnote{https://www.stardog.com/} is a commercial solution that offers a graph database called the "Enterprise Knowledge Graph platform"\cite{Angele2022}. The initial versions only allowed querying their stored data using SPARQL. The support for querying using  GraphQL was added since the release of version 5.1. (Work in progress)
