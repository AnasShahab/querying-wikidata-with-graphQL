\chapter{Conclusion and Future work}
\label{ch:6}
In this report, we introduced knowledge graphs and their practical uses. We highlighted Wikidata, which is a popular knowledge graph, and has many commercial and research oriented applications. We also discussed about RDF and showed how it compares against the Wikidata data model. The report provided a brief understanding of SPARQL and GraphQL, and highlighted some key differences that exist between them. These included the fundamental goals, commercial usage, expressivity and ability to query multiple data sources.

As part of the work, we researched on several different approaches to query RDF data via GraphQL. These included approaches such as Stardog and TopBraid EDG being commercial solutions, and GraphQL-LD and HyperGraphQL being open-sourced approaches. Since GraphQL-LD and HyperGraphQL are open sourced, they provide the opportunity to query arbitrary RDF data source and make modifications to code. We focused on the fundamental principles and features of the two approaches.

Our focus on this report was on querying Wikidata via GraphQL, and hence we used and adapted GraphQL-LD and HyperGraphQL to work with Wikidata accordingly. As they are open-sourced, we were able to make changes that were necessary for the implementation. GraphQL-LD is predicate-based and a workaround is needed to query for resources that are a type of come specific class. We offered three solutions and offered examples for each. Moreover, we produced a default JSON-LD context for GraphQL-LD that can be used when we no context is provided as input. Both GraphQL-LD and HyperGraphQL were used to query data from Wikidata which was demonstrated using examples.

Lastly, we provided a comparison between GraphQL-LD and HyperGraphQL. We highlighted on schema usage, intermediary server, updating data, reverse querying and generated SPARQL queries. When comparing the generated SPARQL queries we used examples that distinguished the capabilities of the two approaches.
GraphQL is more flexible and developer friendly than SPARQL. It is popular in the web development community. However, SPARQL is more expressive and is specialized to query and manipulate RDF data in the Semantic Web. Using GraphQL to query RDF data makes knowledge graphs like Wikidata more convenient to use.  GraphQL-LD and HyperGraphQL provide a layer of abstraction that hides the complexity of SPARQL and allows us to query RDF data using GraphQL.  

Both GraphQL-LD and HyperGraphQL are work in progress. They do not have all the features that are typical to GraphQL. However, they provide a prospective way to query Linked Data via GraphQL. Updating data is crucial when working with knowledge graphs. Mutations are not supported by any of the two approaches and they allow users to only read data from existing RDF datasets. 

Wikidata has different way of representing the relationships between resources as compared to RDF. Analogously, other knowledge graphs can also have such differences based on the vocabulary used. Currently in both GraphQL-LD and HyperGraphQL, we have to modify the source code whenever we want to switch between different knowledge graphs. This is a limitation when we want to query different knowledge graphs and perform federated querying.

Furthermore, the workarounds we proposed for GraphQL-LD to perform subject-based querying can be further developed to make them more user friendly and intuitive. HyperGraphQL does not allow non-null feature in the GraphQL schema. This is a limitation when querying for optional fields. Moreover, passing arbitrary arguments to fields is not supported in HyperGraphQL.
\lukas{Passing arbitrary arguments is generally not supported by GraphQL, they need to be defined in the schema. GraphQL-LD is kind of special here since it has it's own semantics that handles many arguments.}

In the future, we aim to work on updating GraphQL-LD and HyperGraphQL so that they can more efficiently and productively be used to query knowledge graphs. We wish to provide solutions to the existing limitations and incorporate additional features that are available in standard GraphQL specifications. We hope that in doing so we can use the full potential of knowledge graphs like Wikidata. 
