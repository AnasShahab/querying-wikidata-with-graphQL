\chapter{Technicalities and setup}
There were several challenges encountered when trying to test GraphQL-LD and HypergraphQL, and using them to query Wikidata. 

With GraphQL-LD, there were issues running the examples provided in the official documentation. These were related with compatibility when invoking the programmatic API written in ES6 on modules written in CommonJS. To solve this issue we had to update the API codes and the root level node package. Since GraphQL-LD is predicate based, it was challenging to find a workaround to query Wikidata. There were no examples on the documentation for subject-based querying. A significant amount of time and research was invested in finding out the three proposed solutions. The process of creating a default context required a lot of time and resource consumption. The original dump file of truthy statements was around 60GB, and hence required a large bandwidth and memory consumption for downloading and parsing the file. 

The initial problem with HyperGraphQL was with the building of source code using Gradle. After much analyzing and debugging we found there were compatibility issues between versions of Java and Gradle configured. At the time of implementation, the latest version of Gradle was not compatible with the latest version of Java. So we had to downgrade our Java version for it be supported by Gradle. An important part of our work is to analyze the generated SPARQL queries. When working with HyperGraphQL, the viewing of the generated SPARQL queries is not available by default. We had to study the source code in detail in order to log them on the command line. 

We discussed in the previous chapter about the data model in Wikidata where the RDF property name "rdf:type" is replaced by Wikidata property "instance of". This property is used when the subject is an instance of a class. In GraphQL-LD, the fragments in GraphQL queries apply on types. When translating to SPARQL queries, the predicate rdf:type is used to connect the subject to some type of object. We had to analyze the "GraphQL to SPARQL algebra" module in order to update the source code so that GraphQL-LD used "instance of" instead of "rdf:type". Similarly, HyperGraphQL also used rdf:type as the property when translating the root node in GraphQL queries to SPARQL. A similar effort was needed to update the source code such that "instance of" was used instead of "rdf:type" so that data from Wikidata could be queried.

We have created a public repository\footnote{https://github.com/AnasShahab/querying-wikidata-with-graphQL.} where all the changes made to the source code of both the approaches, GraphQL-LD and HyperGraphQL, are retained. This repository can be used to test the querying of Wikidata via both of the approaches. Along with containing the installation guide to use the repository, it contains examples to query Wikidata including the ones used in this report.
