\chapter{Introduction}
%\section{Introduction}

The term "knowledge graph" gained popularity in 2012 when Google launched its own \textit{Google Knowledge Graph}. A knowledge graph is a collection of data represented as a graph. The collected data conveys knowledge of the real world, where the nodes represent entities of interest and edges the many different relations between those entities \cite{Hogan2021}. The entities are real world objects and abstract concepts. For example, "Helium has the chemical formula He", is a statement that can be represented using a knowledge graph. Here the nodes of the graph would represent "Helium" and "He", while the connecting edge between those nodes would represent the relation "chemical formula". 

There are many ways of modelling data as a graph. The most commonly used ones are directed edge-labelled graphs, heterogeneous graphs, property graphs and graph dataset \cite{Hogan2021}. We will see in Section 2 how we can use
\lukas{the? (I get why you did not put it though)} Resource Description Framework (RDF)\footnote{https://www.w3.org/TR/rdf11-concepts/ .} to specify directed edge-labelled graphs.

Many companies such as Amazon, Facebook, Uber, Google, etc., use knowledge graphs for their applications. Depending on the organization or community there are open or enterprise knowledge graphs \cite{Hogan2021}. Open knowledge graphs include Wikidata, DBpedia, Freebase, YAGO, etc. These are available online and freely accessible to the public. Enterprise knowledge graphs are generally used internally within a company and have their commercial specific use-cases \cite{Hogan2021}.
\lukas{do not put this citation after every sentence; maybe after a paragraph}

Wikidata is an open knowledge graph developed by Wikimedia Deutschland. It contains structured data and partly acts as a central database for Wikimedia projects like Wikipedia. Wikidata is built on RDF framework. This enables it to be queried using SPARQL, which is a query language for RDF. However, in commercial applications the use of SPARQL remains limited. One of the main reasons is that developers are not often learned or experienced in the triples that RDF offers, and find the structure of SPARQL queries to be complex. \lukas{this claim cannot be backed by any evidence; this is only speculation and can be phrased as such. Certainly SPARQL is quite complex compared to other query formalism but we don't know if this is the reason for it not being used widely (but it might be). Also I'm not even sure that SPARQL is not used widely. Maybe many applications actually use it.}

GraphQL is a query language popular in commercial applications. It was developed by Facebook in 2012 and made open source in 2015. It is easy to learn and use, providing syntax that is more human friendly than SPARQL. Our goal in this report is to provide a research on ways to query Wikidata using GraphQL, and offer two implementations for this purpose - GraphQL-LD and HyperGraphQL. Both of these are open source and can be used to query arbitrary knowledge graphs using GraphQL.. We also provide comparisons between the implementations, and evaluate their performances along with limitations.
\lukas{this might need to change depending on what will really be part of the evaluation chapter}

\lukas{motivation could still be clearer I think}


The remainder of the report is structured as follows.
\begin{itemize}
	\item In chapter 2, we provide an overview of RDF, Wikidata, SPARQL and GraphQL. We also give a comparison between GraphQL and SPARQL. 
	\item In chapter 3, we show the approaches used to query RDF graphs using GraphQL.
	\item In chapter 4, we aim to provide the implementation of the above approaches on Wikidata. This also contains the technicalities and the differences in the SPARQL queries generated by the tools. Moreover, we also show the differences between the generated SPARQL queries and handwritten SPARQL queries here. Finally, we end the chapterby providing the performance and limitations of the tools.
\end{itemize}

\anas{increase introduction}
